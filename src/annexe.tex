\subsection{Preliminaries}
Firstly recall the more general definition of the \textit{conditional 
expectation} with respect to a random variable:
a random variable is a measurable function defined on the set $\Omega$
equipped with the $\sigma$-algebra $\mathcal{F}$ and the probability measure $\mathbb{P}$.
\begin{Def}[Expectation of $X$ Conditional on $Y$: $\mathbb{E} \croch{ X \sachant Y }$] \label{def_conditional_expectation}
Let $X$ et $Y$ be two random variables defined on $(\Omega,\mathcal{F},\mathbb{P})$:
values of $X$ are in $\mathbb{R}$ equipped with the Borel $\sigma$-algebra $\mathcal{B}(\mathbb{R})$ 
and $X$ is integrable. Values of $Y$ are in a set $\mathcal{Y}$ equipped with a $\sigma$-algebra $\mathcal{V}$.
Let us denote by $\sigma(Y)$ the $\sigma$-algebra generated 
by $Y$ \textit{i.e.} $\sigma(Y) = \set{ Y^{-1}(V) \sachant V \in \mathcal{V}} \subset \mathcal{F}$.

The \textbf{expectation of $X$ conditional on $Y$}, denoted by $\mathbb{E} \croch{ X \sachant Y }$, 
is \textbf{the unique random variable in $\mathbb{L}^1(\Omega,\mathcal{F},\mathbb{P})$ which is} 
\begin{itemize}
\item $\sigma(Y)$-measurable,
\item and such that $\forall A \in \sigma(Y)$, $\displaystyle \int_{A} \mathbb{E} [X \vert Y](\omega) \ d \mathbb{P}(\omega)
 = \int_{A} X(\omega) d \mathbb{P}(\omega)$.
\end{itemize}
As the classical expectation, the conditional expectation is linear: 
if $X_1$ and $X_2$ are two random variables, $\forall c \in \mathbb{R}$,
$\mathbb{E} \croch{ c\cdot X_1 + X_2 \sachant Y  } = c \cdot \mathbb{E} \croch{ X_1 \sachant Y  }  + \mathbb{E} \croch{ X_2 \sachant Y }$.
\end{Def}
First, note that if $X$ is $\sigma(Y)$-measurable, $X = \mathbb{E} \croch{ X \sachant Y }$ $\mathbb{P}$-almost surely.
Indeed, the function $X$ meets both conditions 
to be $ \mathbb{E} \croch{ X \sachant Y }$.
Note also that the second point of Definition \ref{def_conditional_expectation}
implies that \[ \mathbb{E} \Big[ \mathbb{E} \croch{ X \sachant Y } \Big] = \int_{\Omega} \mathbb{E} \croch{ X \sachant Y }(\omega) d \mathbb{P}(\omega) = \int_{\Omega} X(\omega) d \mathbb{P}(\omega) = \mathbb{E} \croch{X}. \] 
This second point 
can be replaced by an other characterization,
given by  the following property:
\begin{Property}
\label{prop_def_condexpec}
The random variable $\mathbb{E} \croch{ X \sachant Y }$ can be defined as
the unique function $\sigma(Y)$-measurable such that
$\forall Z: \mathcal{Y} \rightarrow \mathbb{R}$ $\sigma(Y)$-measurable, 
\[ \mathbb{E} \bigg[ Z \cdot \mathbb{E} \croch{ X \sachant Y } \bigg] = \mathbb{E} \croch{Z \cdot X}. \]
\end{Property}
\begin{proof}
Indeed if $Z = \mathds{1}_{A}$ with $A \in \sigma(Y)$, 
we fall back into the second point of Definition \ref{def_conditional_expectation}. 
Thanks to the linearity of the expectation, 
it remains true if $Z$ 
is a linear combination of characteristic functions of
elements of the $\sigma$-algebra.
Finally, consider a non decreasing sequence $(Z_n)_{n \in \mathbb{N}}$,
with $\forall n \in \mathbb{N}$, $Z_n$ a combination of characteristic functions,
and whose limit is a measurable function $Z$.
Equality holds for each $Z_n$, and thanks to Beppo-Levi Theorem,
the result is true for the measurable function $Z$. 
\end{proof}

This result makes the following property easier to show:
\begin{Property} \label{res_espcond} Let $X$ be a real and integrable random variable, 
and $Y_1,Y_2$ two random variables:
\begin{equation*}
\displaystyle \mathbb{E} \Big[  \mathbb{E} \croch{ X \sachant Y_1,Y_2  } \Big\vert Y_2  \Big] = \mathbb{E} \croch{ X \sachant Y_2  } \hspace{0.15cm} \mathbb{P}\mbox{-almost surely.} 
\end{equation*}
As well,
\[ \displaystyle \mathbb{E} \Big[  \mathbb{E} \croch{ X \sachant Y_2  } \Big\vert Y_1, Y_2  \Big] = \mathbb{E} \croch{ X \sachant Y_2  } \hspace{0.15cm} \mathbb{P}\mbox{-almost surely.} \]
\end{Property}
\begin{proof}
The second equality is obvious because
if $X$ is $\sigma(Y)$-measurable, $\mathbb{E} \croch{ X \sachant Y } = X$ 
$\mathbb{P}\mbox{-almost surely.}$ Indeed, $X$ meets the first condition 
to be $\displaystyle \mathbb{E} \croch{ X \sachant Y }$ in Definition \ref{def_conditional_expectation}, and of course, the second condition is met.
Here, the random variable $\mathbb{E} \croch{ X \sachant Y_2 }$ 
is $\sigma(Y_2)$-measurable by definition,
and thus it is $\sigma(Y_1,Y_2)$-measurable: 
thus the second equality holds.

For the first equality, 
since both conditional expectations are $\sigma(Y_2)$-measurable by definition, 
it is sufficient to show that $\forall Z$ $\sigma(Y_2)$-measurable, 
$\mathbb{E} \bigg[ Z \cdot \mathbb{E} \Big[ \mathbb{E} \croch{ X  \sachant Y_1 , Y_2 }  \Big\vert Y_2  \Big]   \bigg] 
= \mathbb{E} \croch{ Z \cdot X  }$,
as $\mathbb{E} \croch{ X \sachant Y_2 }$ is the only random variable in $\mathbb{L}^1(\Omega,\mathcal{F},\mathbb{P})$ 
such that for each $\sigma(Y_2)$-measurable random variable $Z$,
$\mathbb{E} \croch{ Z \cdot \mathbb{E} \croch{ X \sachant Y_2 } } = \mathbb{E} \croch{ Z \cdot X}$.
Let $Z$ be $\sigma(Y_2)$-mesurable:
as $Z$ is $\sigma(Y_2)$-measurable, 
it is \textit{a fortiori} $\sigma(Y_1,Y_2)$-measurable: 
$\sigma(Y_2) \subseteq \sigma(Y_1,Y_2)$.
Thus 
\[ \mathbb{E} \bigg[ Z \cdot \mathbb{E} \Big[ \mathbb{E} \croch{ X \sachant Y_1 , Y_2 } \Big\vert Y_2  \Big] \bigg] 
= \mathbb{E} \Big[ Z \cdot \mathbb{E} \croch{ X  \sachant Y_1 , Y_2 } \Big]
= \mathbb{E} \croch{ Z \cdot X }. \] 
\end{proof}



The conditional expectation has been defined 
as a random variable $\mathbb{E} \croch{ X \sachant Y }: \Omega \rightarrow \mathbb{R}$. 
However, the following property implies that
the $\sigma(Y)$-measurability condition of Definition \ref{def_conditional_expectation} 
may be replaced by 
``$\exists \varphi: (\mathcal{Y}, \mathcal{V}) \rightarrow \Big( \mathbb{R}, \mathcal{B}(\mathbb{R})\Big)$  measurable 
such that $\mathbb{E} \croch{ X \sachant Y  } = \varphi (Y)$''.
The function $\varphi$ is called \textit{Expectation of $X$ Conditional on the Values
of $Y$} and may be denoted by $\varphi(y) = \mathbb{E} \croch{ X \sachant Y = y}$,
$\forall y \in \mathcal{Y}$. 
\begin{Property}
\label{sigmaYmeas_phi}
The function $Z:\Omega \rightarrow \mathbb{R}$ is $\sigma(Y)$-measurable, where $Y: (\Omega,\mathcal{F}) \rightarrow (\mathcal{Y},\mathcal{V})$ \\ 
$\Leftrightarrow$ $\exists \varphi: (\mathcal{Y}, \mathcal{V}) \rightarrow \Big( \mathbb{R}, \mathcal{B}(\mathbb{R}) \Big)$  measurable such that $Z = \varphi(Y)$.
\end{Property}
\begin{proof}
The set $\set{ \varphi(Y) \sachant \varphi: (\mathcal{Y}, \mathcal{V}) \rightarrow \paren{ \mathbb{R}, \mathcal{B}(\mathbb{R})} \mbox{ measurable }  }$ is denoted by $\Phi$.
If $Z \in \Phi$, then $Z = \varphi(Y)$ with $\varphi$ measurable,
and as $Y$ is $\sigma(Y)$-measurable, $Z$ is $\sigma(Y)$-measurable: 
in short, if $Z \in \Phi$, then $Z$ is $\sigma(Y)$-measurable. 

Now let us show that any $\sigma(Y)$-mesurable function 
can be written $\varphi(Y)$ with $\varphi$ measurable. 
By definition $\forall A \in \sigma(Y)$, $\exists B \in \mathcal{V}$ such that 
$A = \set{ Y \in B } = Y^{-1}(B)$. Thus $\mathds{1}_{A} = \mathds{1}_{ \set{Y \in B}} = \mathds{1}_B(Y)$ is in $\Phi$
as a function of $Y$.
Now, as linear combinations of such characteristic functions are in $\Phi$,
and as non-decreasing limits of sequences of functions in $\Phi$ are in $\Phi$,
it can be concluded that $\Phi$ contains all $\sigma(Y)$-measurable functions.
\end{proof}
\begin{Def}[Expectation of $X$ Conditional on the Values of $Y$]
Let $X \in \mathbb{R}$ and $Y \in \mathcal{Y}$ two random variables:
as $\mathbb{E} \croch{ X \sachant Y }$ can be written $\varphi(Y)$
with $\varphi:\mathcal{Y} \rightarrow \mathbb{R}$ measurable, 
\[ \mathbb{E} \Big[ \ X \ \Big\vert \  Y = y \ \Big] = \varphi(y) \]
is called the expectation of $X$ conditional on the event $\set{Y=y}$. 
For each $y \in \mathcal{Y}$ such that $\mathbb{P} \paren{ Y = y}>0$, 
it is the expectation of $X$ 
if we know that $Y=y$.

If $\mathcal{Y}$ is countable, and $\mathbb{P} \paren{ Y = y}>0$,
\[ \mathbb{E} \croch{ X \sachant Y = y } 
%= \int_{\set{Y=y}} \frac{\varphi(Y)}{\mathbb{P}(Y=y)} d \mathbb{P} 
=  \int_{\set{Y=y}} \frac{\mathbb{E} \croch{ X \sachant Y }(\omega)}{\mathbb{P}(Y=y)} d \mathbb{P}(\omega) .\]
\end{Def}
Consider that the values of variables $X$ and $Y$ are in $\mathcal{S}$, 
and let us introduce $f:\mathcal{S} \rightarrow \mathbb{R}$ measurable:
$f(X)$ is a random variable whose values are in $\mathbb{R}$.
The expectation of $f(X)$ conditional on the values of $Y \in \mathcal{S}$ 
is denoted by $\mathbb{E} \croch{ f(X) \sachant Y = y }$,
$\forall y \in \mathcal{S}$.
If $\mathcal{S}$ is a countable set 
equipped with the $\sigma$-algebra $\mathcal{P}(\mathcal{S})$ 
(the set of sets included in $\mathcal{S}$),
$\varphi(y) = \mathbb{E} \croch{ f(X) \sachant Y = y }$ 
can be computed explicitly.
\vbox{
\begin{Property}
Let $X$ and $Y$ two random variables whose values are in a countable set $\mathcal{S}$,
and $f: \mathcal{S} \rightarrow \mathbb{R}$ a measurable function:
the expectation of $f(X)$ conditional on the values of $Y$ is
\begin{equation*}  
\forall y \in \mathcal{S} \mbox{ such that } \mathbb{P} \paren{ Y = y }>0, \ \ \ \mathbb{E} \Big[ f(X) \Big\vert Y=y \Big]=  \sum_{x \in \mathcal{S}} f(x) \cdot \mathbb{P} \paren{ X=x \sachant Y=y }.
\end{equation*}
where $\mathbb{P} \paren{ X=x \sachant Y=y } %= \mathbb{E} \croch{ \mathds{1}_{\set{X=x}} \sachant Y = y } 
= \frac{\mathbb{P} \paren{ X=x, Y=y }}{\mathbb{P} \paren{ Y=y }}$.
\end{Property}
}
\begin{proof}
For clarity, $\mathbb{E} \croch{ f(X) \sachant Y=y }$
is denoted by $\varphi(y)$ in this proof.
By only considering singletons $\set{ Y = y } \subseteq \Omega$ 
in the second condition of Definition \ref{def_conditional_expectation},
it leads to $ \forall y \in \mathcal{S}$,
\[ \int_{ \set{ Y = y } } \varphi(Y) d \mathbb{P} = \int_{  \set{ Y = y }  } f(X) d \mathbb{P} \ \ \Leftrightarrow \ \ \varphi(y) \cdot \mathbb{P} \paren{Y=y} = \int_{ \set{Y=y}} f(X) d \mathbb{P}\]
and if $\mathbb{P} \paren{ Y=y }>0$,	
\[\varphi(y) = \int_{\Omega} f(X) \cdot  \dfrac{ \mathds{1}_{\set{Y=y}} d \mathbb{P}}{ \mathbb{P} \paren{ Y = y } } =  \int_{\Omega} \sum_{x \in \mathcal{S}} f(x) \cdot \mathds{1}_{\set{X=x}} \cdot \dfrac{ \mathds{1}_{\set{Y=y}} d \mathbb{P}}{ \mathbb{P} \paren{ Y=y  } } =  \sum_{x \in \mathcal{S}} f(x) \cdot \dfrac{ \mathbb{P} \paren{ X = x, Y = y } }{ \mathbb{P} \paren{ Y = y } } \]
thanks to the Fubini theorem. 
Note that if $f(X) = \mathds{1}_{\set{X=x}}$, 
we get $\mathbb{E} \croch{ \mathds{1}_{ \set{X=x}} \sachant Y=y } = \dfrac{ \mathbb{P} \paren{ X = x, Y = y } }{ \mathbb{P} \paren{ Y = y }}$,
which is the discrete conditional probability $\mathbb{P} \paren{ X = x \sachant Y = y }$.
\end{proof}


\subsection{Proof of Property \ref{res_markov}}
\label{res_markov_RETURN}
\begin{proof}
Let $t \in \mathbb{N}$, $(s_0,s_1,\ldots,s_t) \in \mathcal{S}^t$ and $s' \in \mathcal{S}$: 
on the one hand,
\begin{equation*}
\int_{ \set{ S_0=s_0,\ldots,S_t=s_t } } \mathds{1}_{ \set{S_{t+1} =s'}} d \mathbb{P} = \mathbb{P} \paren{ S_0=s_0,\ldots,S_t=s_t,S_{t+1}=s' },
\end{equation*}
and on the other hand, if $\mathbb{E} \croch{ \mathds{1}_{\set{S_{t+1} = s'}} \sachant S_t }$ 
is denoted by $\mathbb{P} \paren{ S_{t+1} = s' \sachant S_t }$,
\[ \displaystyle \int_{ \set{ S_0=s_0,\ldots,S_t=s_t } } \mathbb{P} \paren{ S_{t+1} = s' \sachant S_t } d \mathbb{P} 
= \mathbb{P} \paren{ S_{t+1} = s' \sachant S_t=s_t } \cdot \mathbb{P} \paren{ S_0=s_0,S_1=s_1,\ldots,S_t=s_t }.\]
Both integrals are equal as $(S_t)_{t \in \mathbb{N}}$ is a Markov Chain. 
Since events of $\sigma(S_0,\ldots,S_t)$ are unions of events which can be written $ \set{S_0=s_0,S_1=s_1,\ldots,S_t=s_t}$, 
for all $B \in \sigma \paren{ S_0,\ldots,S_t }$
\[ \int_B \mathds{1}_{ \set{S_{t+1} =s'}} d \mathbb{P} = \int_B  \mathbb{P}  \paren{ S_{t+1} = s' \sachant S_t }  d \mathbb{P}  \]
and then, using Definition \ref{def_conditional_expectation} and as 
$ \mathbb{P}  \paren{ S_{t+1} = s' \sachant S_t }$ is $\sigma(S_0,\ldots,S_t)$-measurable
\[ \mathbb{P}  \paren{ S_{t+1} = s' \sachant S_t } = \mathbb{E} \croch{ \mathds{1}_{S_{t+1} =s'} \sachant S_0,\ldots,S_t  } \ \mathbb{P}\mbox{-almost surely.} \]
Now, 
\begin{eqnarray*}
\mathbb{E} \croch{ f(S_{t+1}) \sachant S_0,\ldots,S_t } 
& = & \mathbb{E} \croch{ \sum_{s' \in \mathcal{S}} f(s') \cdot \mathds{1}_{ \set{S_{t+1}=s'} }  \sachant S_0,\ldots,S_t } \\
& = & \sum_{s' \in \mathcal{S}} f(s') \cdot \mathbb{E} \croch{ \mathds{1}_{\set{S_{t+1}=s'}}  \sachant S_0,\ldots,S_t } \\
& = & \sum_{s' \in \mathcal{S}} f(s') \cdot \mathbb{P} \paren{ S_{t+1}=s'  \sachant S_t } \ \ \mathbb{P}\mbox{-almost surely.} %\\
\end{eqnarray*}
The random variable $\sum_{s' \in \mathcal{S}} f(s') \cdot \mathbb{P} \paren{ S_{t+1}=s'  \sachant S_t }$ 
is $\sigma(S_t)$-measurable, thus $\mathbb{E} \croch{ f(S_{t+1}) \sachant S_0,\ldots,S_t } $
is $\sigma(S_t)$-measurable too, and then,
\[ \mathbb{E} \croch{ f(S_{t+1}) \sachant S_0,\ldots,S_t }  = \mathbb{E} \croch{ \mathbb{E} \croch{ f(S_{t+1}) \sachant S_0,\ldots,S_t } \hspace{-0.23cm} {\color{white} \int} \sachant S_t } = \mathbb{E} \croch{ f(S_{t+1}) \sachant S_t },  \]
because of Property \ref{res_espcond}.
Finally, the equalities of Property \ref{res_markov} are achieved 
by integrating both parts of the equations 
over $\set{S_0 = s_0, \ldots, S_t = s_t}$
and then by dividing them 
by $\mathbb{P} \paren{ S_0 = s_0, \ldots, S_t = s_t }$.
\end{proof}







\subsection{Proof of Theorem \ref{thm_mdp_finiteH}}
\label{thm_mdp_finiteH_RETURN}
\begin{proof}
First of all, consider that the process is at the stage $\tilde{t} \in \mathbb{N}$.
All states from the beginning of the process are given as input to the agent:
it has to choose the best action knowing these $\tilde{t}+1$ first system states 
$\set{ s_0, s_1, \ldots, s_{\tilde{t}} } \in \mathcal{S}^{\tilde{t}+1}$.
Regardless the previously gathered rewards, 
its goal is to maximize the expectation of the sum of the next rewards. 
We show by induction on $\tilde{t}$ from $H-1$ to $0$,
that the highest expected total reward can be reached with a strategy $(d_t)_{t=0}^{H-1}$
\textit{i.e.} a sequence of decision rules $d_t: \mathcal{S} \rightarrow \mathcal{A}$.
A sequence of functions from all the states that the system has gone through,
\textit{i.e.} a sequence $(d_t)_{t=0}^{H-1}$ of functions $d_t: \mathcal{S}^t \rightarrow \mathcal{A}$, 
is not necessary.

Let $\tilde{t} = H-1$: the agent has to find the action $a$ maximizing
\[ \mathbb{E} \Big[ \ r_{H-1}(S_{H-1},a) + R(S_H) \ \Big\vert \ S_0=s_0, \ldots, S_{H-1} = s_{H-1} \ \Big] \]
which is equal to 
$\displaystyle \mathbb{E} \Big[ \ r_{H-1}(S_{H-1},a) 
+ \mathbb{E} \croch{ R(S_H) \sachant S_{H-1} } \ \Big\vert \ S_0=s_0, \ldots, S_{H-1} = s_{H-1} \ \Big]$
\begin{align*}
& = \mathbb{E} \croch{ f_a(S_{H-1}) \sachant S_0=s_0, \ldots, S_{H-1} = s_{H-1} } 
\mbox{ with } f_a: \mathcal{S} \rightarrow \mathbb{R} \mbox{ measurable,}  \\ 
& = \mathbb{E} \croch{ f_a(S_{H-1}) \sachant S_{H-1} = s_{H-1}}% = V_{1}^a(s_{H-1})
\end{align*}
because of Property (\ref{res_markov}).
Then, the value to be maximized depends on the state $s_{H-1} \in \mathcal{S}$ only:
a decision rule $d^*_{H-1}: \mathcal{S} \rightarrow \mathcal{A}$ 
such that $\forall s \in \mathcal{S}$
\[ d^*_{H-1}(s) \in \operatorname*{argmax}_{a \in \mathcal{A}} \mathbb{E} \croch{ f_a(S_{H-1}) \sachant S_{H-1} = s} \] is sufficient.
Now, assume that this result is true for the time step $\tilde{t}+1 \leqslant H-1$,
and consider that a strategy $(d_t)_{t=\tilde{t}+1}^{H-1}$ has been computed.
The agent has to find the action $a$ maximizing
\[ \mathbb{E} \croch{ r_{\tilde{t}}(S_{\tilde{t}},a) + \sum_{t = \tilde{t}+1}^{H-1} r_{t}\Big(S_{t},d_t(S_t)\Big) + R(S_H) \sachant S_0=s_0, \ldots, S_{\tilde{t}} = s_{\tilde{t}} } \]
which is equal to
$\mathbb{E} \Bigg[ \ r_{\tilde{t}}(S_{\tilde{t}},a) + \mathbb{E} \croch{ \sum_{t = \tilde{t}+1}^{H-1} r_t\Big(S_{t},d_t(S_t)\Big) + R(S_H) \sachant  S_{\tilde{t}} } 
\ \bigg\vert \ S_0=s_0, \ldots, S_{\tilde{t}} = s_{\tilde{t}} \ \Bigg]$
\begin{align*}
\hspace{3cm} & = \mathbb{E} \croch{ f_a(S_{\tilde{t}}) \sachant S_0=s_0, \ldots, S_{\tilde{t}} = s_{\tilde{t}} } 
\mbox{ with } f_a: \mathcal{S} \rightarrow \mathbb{R} \mbox{ measurable,}  \\ 
& = \mathbb{E} \croch{ f_a(S_{\tilde{t}}) \sachant S_{\tilde{t}} = s_{\tilde{t}}} 
\end{align*}
and then, the same conclusion holds:
it is sufficient to compute a decision rule 
$d^*_{\tilde{t}}: \mathcal{S} \rightarrow \mathcal{A}$ 
such that $\forall s \in \mathcal{S}$
\[ d^*_{\tilde{t}}(s) \in \operatorname*{argmax}_{a \in \mathcal{A}} \mathbb{E} \croch{ f_a(S_{\tilde{t}}) \sachant S_{\tilde{t}} = s}. \]

We just proved by induction that it suffices to look for
strategies of $\mathcal{D}_H$ as defined above, 
\textit{i.e.} to look for a sequence of decision rules $(d_t)_{t=0}^{H-1}$,
maximizing the criterion \ref{criterion}.

In the following, we set up the Dynamic Programming equations
used to compute the optimal value function and the optimal strategy.
The size of the horizon $i$ is the index used for the incremental computation 
of the optimal value function $V^*$. 
It is also the opposite modulo $H$ of the stage of the process $t$,
index used for the strategy. 
The initialization $V^*_0(s)=R(s)$ is obvious: 
when the horizon is zero, no action has to be selected by the agent,
and it receives only the terminal reward. 
Next, let $s \in \mathcal{S}$, $i \in \set{1,\ldots H}$, 
and set $t_0 = H-i$:
\begin{eqnarray*}
V^*_{i}(s) & = & \sup_{(d_{t})_{t=t_0}^{H-1} \in \mathcal{D}_{H-t_0}} 
\mathbb{E} \croch{ \sum_{t=t_0}^{H-1} r_t\Big(S_t,d_t(S_t)\Big) + R(S_H) \sachant S_{t_0}=s }.
\end{eqnarray*}
Yet $ \displaystyle \mathbb{E} \croch{ \sum_{t=t_0}^{H-1} r_t\Big(S_i,d_i(S_i)\Big) + R(S_H) \sachant S_{t_0} } $
\begin{eqnarray*}
& = & \mathbb{E} \croch{ r_{t_0}\Big(S_{t_0},d_{t_0}(S_{t_0})\Big) 
+ \mathbb{E} \croch{  \sum_{t=t_0+1}^{H-1} r_t\Big(S_i,d_i(S_i)\Big) + R(S_H) \sachant S_{t_0},S_{t_0+1} } \sachant S_{t_0} } \\
& = & \mathbb{E} \bigg[ r_{t_0}\Big(S_{t_0},d_{t_0}(S_{t_0})\Big) 
+ \mathbb{E} \croch{  V_{i-1}\Big(S_{t_0+1},(d_t)_{t=t_0+1}^{H-1}\Big) \sachant S_{t_0},S_{t_0+1} } \bigg\vert S_{t_0} \bigg] \\
& = & \mathbb{E} \croch{ r_{t_0}\Big(S_{t_0},d_{t_0}(S_{t_0}) \Big) 
+ \sum_{s' \in \mathcal{S}_{s,a,t}} \textbf{p}_{t_0} \Big( s' \Big\vert S_{t_0} , d_{t_0}(S_{t_0}) \Big) \cdot V_{i-1}\Big(s',(d_t)_{t=t_0+1}^{H-1}\Big) \sachant S_{t_0} } \\
\end{eqnarray*}
using properties (\ref{res_espcond}) and (\ref{res_markov}).
By integrating both part of this equality over $\set{ S_{t_0}=s }$ 
and divising them by $\mathbb{P} \paren{ S_{t_0}=s } > 0$,
it becomes
\begin{eqnarray}
\nonumber V^*_{i}(s) & = & \sup_{(d)_{t=t_0}^{H-1} \in \mathcal{D}_{H - t_0} } \set{  r_t \Big( s,d_{t_0}(s) \Big) 
+ \sum_{s' \in \mathcal{S}_{s,a,t}} \textbf{p}_{t_0} \Big( s' \Big\vert s , d_{t_0}(s) \Big) \cdot V_{i-1} \Big( s',(d_t)_{t=t_0+1}^{H-1} \Big) } \\
\nonumber & = & \sup_{ \big(a,(d')_{t=t_0+1}^{H-1}\big) \in \mathcal{D}_{H-t_0} } \set{ r_t(s,a) 
+ \sum_{s' \in \mathcal{S}_{s,a,t}} \textbf{p}_{t_0} \paren{s' \sachant s , a } \cdot V_{i-1} \Big( s',(d_t') \Big) } \\
\label{munos} & = & \max_{a \in \mathcal{A} } \set{ r_t(s,a) 
+ \sum_{s' \in \mathcal{S}_{s,a,t}} \textbf{p}_{t_0} \paren{s' \sachant s , a } \cdot \sup_{(d_{t}')_{t=t_0+1}^{H-1} \in \mathcal{D}_{H-t_0-1}}  V_{i-1} \Big( s',(d_t') \Big) } \\
\nonumber & = & \max_{ a \in \mathcal{A} } \set{ r_t(s,a) 
+ \sum_{s' \in \mathcal{S}_{s,a,t}} \textbf{p}_{t_0} \paren{s' \sachant s , a } \cdot V_{i-1}^* \paren{ s' } }
\end{eqnarray}
where \ref{munos} is justified by:
\begin{itemize}
\item[$\bullet$] as $ \displaystyle \sum_{s' \in \mathcal{S}_{a,s,t}} \textbf{p}_{t_0} \paren{s' \sachant s , a } \cdot V_{i-1} \Big( s',(d_t) \Big) \leqslant  \sum_{s' \in \mathcal{S}_{a,s,t}} \textbf{p}_{t_0} \paren{s' \sachant s , a } \sup_{(d_{t}') \in \mathcal{D}_{H-t_0-1} } V_{i-1} \Big( s',(d_t') \Big)$
for each strategy $(d_t)_{t=t_0 +1}^{H-1} \in \mathcal{D}_{H-t_0-1}$,
\[ \sup_{(d_{t}') \in \mathcal{D}_{H-t_0-1} } \sum_{s' \in \mathcal{S}_{a,s,t}} \hspace{-0.2cm} \textbf{p}_{t_0} \paren{s' \sachant s , a } V_{i-1} \Big( s',(d_t') \Big) \leqslant 
 \sum_{s' \in \mathcal{S}_{a,s,t}} \hspace{-0.2cm}
\textbf{p}_{t_0} \paren{s' \sachant s , a } \sup_{(d_{t}') \in \mathcal{D}_{H-t_0-1} } \hspace{-0.1cm} V_{i-1}^* (s'). \]
\item[$\bullet$] let $(\varepsilon_n)_{n \in \mathbb{N}}$ be a sequence of positive real numbers, 
such that $\varepsilon_n \rightarrow 0$ when $n \rightarrow \infty$. 
For each $n \in \mathbb{N}$, let $(d^{\varepsilon_n}_t)_{t=t_0+1}^{H-1} \in \mathcal{D}_{H-t_0-1}$ 
be a strategy such that \[ V^*_{i-1}(s') - \varepsilon \leqslant V_{i-1}\Big(s', (d^{\varepsilon_n})\Big) \leqslant V_{i-1}^*(s'), \ \forall s' \in \mathcal{S}. \]
Computing the mean with respect to $\textbf{p}_{t_0} \paren{ . \sachant s,a}$ whose support is finite,
the inequality on the left becomes
\[ \sum_{s' \in \mathcal{S}_{s,a,t}} \textbf{p}_{t_0} \paren{ s' \sachant s,a} \cdot V^*_{i-1}(s') -\varepsilon_n \leqslant \sum_{s' \in \mathcal{S}_{s,a,t}} \textbf{p}_{t_0} \paren{ s' \sachant s,a} \cdot V_{i-1}\Big(s', (d^{\varepsilon_n})\Big) \] 
where the right part is obviously lower than $ \displaystyle \sup_{(d) \in \mathcal{D}_{H-t_0-1}} \sum_{s' \in \mathcal{S}_{s,a,t}} \hspace{-0.1cm} \textbf{p}_{t_0} \paren{ s' \sachant s,a} \cdot V_{i-1}\Big(s', (d)\Big) $.
Now, as this couple of inequalities are true 
for each $n \in \mathbb{N}$, 
making $n \rightarrow \infty$, the result is
\[ \sum_{s' \in \mathcal{S}_{s,a,t}} \textbf{p}_{t_0} \paren{ s' \sachant s,a} \cdot V^*_{i-1}(s')  \leqslant \sup_{(d) \in \mathcal{D}_{H-t_0-1}} \sum_{s' \in \mathcal{S}_{s,a,t}} \textbf{p}_{t_0} \paren{ s' \sachant s,a} \cdot V_{i-1}\Big(s', (d)\Big). \] 
\end{itemize}
Therefore, if at each iteration $i \in \set{ 1, \ldots, H }$,
the decision rule $d^*_{H-i}$ is defined as $\forall s \in \mathcal{S}$,
$d_{H-i}^*(s) \in \operatorname*{argmax}_{a \in \mathcal{A}} \set{ r_t(s,a) 
+ \sum_{s' \in \mathcal{S}_{s,a,t}} \textbf{p}_{t_0} \paren{s' \sachant s , a } \cdot V_{i-1}^* \paren{ s' } }$, $V^*_i \Big( s, (d^*_t)_{t = H-i}^{H-1} \Big) = V^*_i(s)$:
with $i = H$, it shows that $(d^*_t)_{t=0}^{H-1} \in \mathcal{D}_H$ is optimal.
\end{proof}








\subsection{Proof of the Bellman Equation (\ref{equation_bellman})}
\label{equation_bellman_RETURN}
\begin{proof}
The following calculus lines 
lead to the Bellman Equation.
Let $(d)_{t \in \mathbb{N}} \in \mathcal{D}_{\infty}$.
\begin{eqnarray}
\nonumber  V^{d}(s) &  := & \mathbb{E} \croch{ \sum_{t=0}^{+\infty} \gamma^{t} \cdot r\Big(S_{t},d_t(S_{t})\Big) \sachant S_0=s}  \\
\nonumber & = & \mathbb{E} \croch{ r\Big(S_0,d_0(S_0)\Big) + \sum_{t=1}^{+\infty} \gamma^{t} \cdot r\Big(S_{t},d_t(S_{t})\Big) \sachant S_0=s} \\
\nonumber & = & r\Big(s,d_0(s)\Big) + \mathbb{E} \croch{ \mathbb{E} \bigg[ \sum_{t=1}^{+\infty} \gamma^{t} \cdot r\Big(S_{t},d_t(S_{t})\Big) \bigg\vert S_1 \bigg] \sachant S_0=s  } \\
\nonumber & = & r\Big(s,d_0(s)\Big) + \gamma \cdot \sum_{s' \in \mathcal{S}_{s,d_0(s)}} \textbf{p} \Big(s' \Big\vert s,d_0(s) \Big) \cdot  \mathbb{E} \croch{ \sum_{t=1}^{+\infty} \gamma^{t-1} \cdot r\Big(S_{t},d_t(S_{t})\Big) \sachant S_1=s' }  \\
\nonumber & = & r\Big(s,d_0(s)\Big) + \gamma \cdot \sum_{s' \in \mathcal{S}_{s,d_0(s)}} \textbf{p} \Big( s' \Big\vert s,d_0(s) \Big) \cdot  \mathbb{E} \croch{ \sum_{t'=0}^{+\infty} \gamma^{t'} \cdot r\Big(S^+_{t'},d^+_{t'}(S^+_{t'})\Big) \sachant S^+_0=s' } \\
\nonumber & = & r\Big(s,d_0(s)\Big) + \gamma \cdot \sum_{s' \in \mathcal{S}_{s,d_0(s)}} \textbf{p} \Big(s' \Big\vert s,d_0(s)\Big)  \cdot V^{d^+}(s').
\end{eqnarray}
The third and fourth lines come from the properties \ref{res_espcond} and \ref{res_markov}.
In the fifth line, $(S^+_t)_{t \in \mathbb{N}}$ is defined as
$S^+_t = S_{t+1}$. 
As well, $\forall t \in \mathbb{N}$, 
$\forall s \in \mathcal{S}$,
$d^+_t(s) = d_{t+1}(s)$. 
\end{proof}






\subsection{Proof of Theorem \ref{FB_complete}}
\label{FB_complete_RETURN}
\begin{proof} 
Let $(V_n)_{n \in \mathbb{N}}$ be a Cauchy sequence 
in $\paren{ \mathcal{F}_{B}(\mathcal{S},\mathbb{R}), \normsup{.}}$. 
For each state $s \in \mathcal{S}$, 
$\Big(V_n(s)\Big)_{n \in \mathbb{N}}$
is a Cauchy sequence of $(\mathbb{R},\abs{.})$
because $\abs{V_n(s)} \leqslant \normsup{V_n}$. 
Since $(\mathbb{R},\abs{.})$ is complete, 
$\forall s \in \mathcal{S}$, $\Big(V_n(s)\Big)_{n \in \mathbb{N}}$ 
has a limit in $\mathbb{R}$
that we denote by $V(s)$. 
A Cauchy sequence is bounded, 
therefore $\exists M>0$ 
such that $\forall n \in \mathbb{N}$, 
$\normsup{V_n} \leqslant M $. 
Thus, $\forall s \in \mathcal{S}$, 
$\abs{ V_n(s) } \rightarrow \abs{ V(s) } \leqslant M$, 
and then $V \in \mathcal{F}_{B}(\mathcal{S},\mathbb{R})$.
Finally, as $(V_n)$ is a Cauchy sequence,
\begin{eqnarray*}
\forall \varepsilon >0, \exists N \geqslant 0, \mbox{ such that } &  \forall n \geqslant N,p \geqslant 0,  \forall s \in \mathcal{S}, & \abs{ V_n(s) - V_{n+p}(s) } < \varepsilon \\
\Rightarrow & \forall n \geqslant N, \forall s \in \mathcal{S}, & \abs{ V_n(s) - V(s) } < \varepsilon \\
\Rightarrow & \forall n \geqslant N, & \normsup{V_n - V} < \varepsilon 
\end{eqnarray*}
Thus $V_n \longrightarrow V$ in $\mathcal{F}_{B}(\mathcal{S},\mathbb{R})$ 
when $n \rightarrow + \infty$, 
and then $\paren{\mathcal{F}_{B}(\mathcal{S},\mathbb{R}),\normsup{.}}$ is a Banach space.
\end{proof}



\subsection{Proof of Property \ref{bellman_contracting}}
\label{bellman_contracting_RETURN}
\begin{proof}
Let $(V,V') \in \mathcal{F}_{B} (\mathcal{S},\mathbb{R})^2$,
\begin{eqnarray*}
\normsup{\mathcal{B}^d V - \mathcal{B}^d V'} & \leqslant &  \gamma \cdot \sup_{s \in \mathcal{S}} \sum_{s' \in \mathcal{S}_{s,d_0(s)}} \textbf{p} \Big( s' \Big\vert s,d_0(s) \Big) \cdot \abs{ V(s')-V'(s')} \\
& \leqslant &  \gamma \cdot \sup_{s \in \mathcal{S}} \sum_{s' \in \mathcal{S}_{s,d_0(s)}} \textbf{p} \Big(s' \Big\vert s,d_0(s) \Big) \cdot \normsup{ V-V'} \\ 
& \leqslant &  \gamma \cdot \normsup{ V-V'} 
\end{eqnarray*}
since $\forall s \in \mathcal{S}$, $\textbf{p} \paren{. \sachant s,d_0(s)}$
is a probability distribution.
\end{proof}



\subsection{Proof of Property \ref{property_DPcontract}}
\label{property_DPcontract_RETURN}
First, let us present the following result:
\begin{Property}
Let $f$ and $g$ be two functions defined on the finite set $\mathcal{A}$ 
and with values in $\mathbb{R}$:
\begin{equation}
\label{res_max}
 \abs{\max_{a \in \mathcal{A}}f(a) - \max_{a \in \mathcal{A}}g(a)} \leqslant \max_{a \in \mathcal{A}} \abs{f(a)-g(a)}
\end{equation}
\label{contraction_max}
\end{Property}
\begin{proof}
\begin{eqnarray*}
& \forall a \in \mathcal{A}, & f(a) - g(a) \leqslant \max_{a' \in \mathcal{A}} \abs{ f(a') - g(a') } \\
 \Rightarrow & \forall a \in \mathcal{A}, & f(a) \leqslant \max_{a' \in \mathcal{A}} g(a') + \max_{a' \in \mathcal{A}} \abs{f(a')-g(a')}  \\ 
 \Rightarrow &  & \max_{a' \in \mathcal{A}} f(a') - \max_{a' \in \mathcal{A}} g(a') \leqslant \max_{a' \in \mathcal{A}} \abs{f(a')-g(a')}.
\end{eqnarray*}
Finally, the same inequalities hold starting with $g(a) - f(a)$, 
thus we get the result \ref{res_max}.
\end{proof}
Here is the proof of Property \ref{property_DPcontract}:
\begin{proof}
The contraction inequality of the operator $\mathcal{B}^d$ for $\paren{V,V'} \in \mathcal{F}_{B} (\mathcal{S},\mathbb{R})^2$
is true for each strategy $(d) \in \mathcal{D}_{\infty}$, 
as stated by Property \ref{bellman_contracting}:
\begin{eqnarray*}
& \forall (d) \in \mathcal{D}_{\infty}, & \normsup{\mathcal{B}^d V - \mathcal{B}^d V'}   \leqslant \gamma \cdot \normsup{V-V'} \\
\Rightarrow & \forall (d) \in \mathcal{D}_{\infty}, \forall s \in \mathcal{S}, & \abs{(\mathcal{B}^{d} V)(s) - (\mathcal{B}^d V')(s)} \leqslant \gamma \cdot \normsup{V-V'} \\
\Rightarrow & \forall a \in \mathcal{A}, \forall s \in \mathcal{S}, & \abs{(\mathcal{B}^aV)(s) - (\mathcal{B}^aV')(s)} \leqslant \gamma \cdot \normsup{V-V'} \\
\Rightarrow &  \forall s \in \mathcal{S}, & \max_{a \in \mathcal{A}} \abs{(\mathcal{B}^aV)(s) - (\mathcal{B}^aV')(s)} \leqslant \gamma \cdot \normsup{V-V'}
\end{eqnarray*}
and Property \ref{contraction_max} leads then to $\forall s \in \mathcal{S}$ $ \displaystyle \abs{ \max_{a \in \mathcal{A}}(\mathcal{B}^aV)(s) - \max_{a \in \mathcal{A}} (\mathcal{B}^aV')(s)} \leqslant \gamma \cdot \normsup{V-V'} $ \\
\[\Rightarrow \normsup{\mathcal{B}^*V-\mathcal{B}^*V'} \leqslant \gamma \cdot \normsup{V-V'}  \]
\end{proof}





\subsection{Proof of Theorem \ref{vstar_optimal}}
\label{vstar_optimal_RETURN}
\begin{proof}
Here is how we show that the function $V^*$ is the optimal value function, 
\textit{i.e.} 
\[ V^*(s) = \sup_{d \in \mathcal{D}_{\infty}} V^d(s) =\sup_{d \in \mathcal{D}_{\infty}} \mathbb{E} \croch{\sum_{t=0}^{+ \infty} \gamma^t \cdot r\Big(S_t,d_t(S_t)\Big) \sachant S_0=s }. \]
Let $s \in \mathcal{S}$ and $d=(a,d^+)$ 
the strategy which consists in selecting action $a$ 
at time step $t=0$, 
and then actions $a_1=d^+_0(s_1):=d_1(s_1),a_2=d^+_1(s_2) := d_2(s_2)$, etc. 
Note that $d^+ \in \mathcal{D}_{\infty}$ as well,
and numbered from $0$. 
\begin{eqnarray}
\nonumber \sup_{(d) \in \mathcal{D}_{\infty}} V^d(s) & := & \sup_{(d) \in \mathcal{D}_{\infty}} \mathbb{E} \croch{\sum_{t=0}^{+ \infty} \gamma^t \cdot r\Big(S_t,d_t(S_t)\Big) \sachant S_0=s } \\
\nonumber & = & \sup_{(a,d^+) \in \mathcal{D}_{\infty}} \set{ r(s,a) + \gamma \cdot \sum_{s' \in \mathcal{S}_{s,a}} \textbf{p} \paren{s' \sachant s,a} \cdot V^{d^+}(s')} \\
\label{justif_bellopt} & = & \max_{a \in \mathcal{A}} \set{ r(s,a) + \gamma \cdot \sum_{s' \in \mathcal{S}_{s,a}} \textbf{p} \paren{s' \sachant s,a} \cdot \sup_{(d^+) \in \mathcal{D}_{\infty}} V^{d^+}(s')}  \\
\nonumber & = & \max_{a \in \mathcal{A}} \set{ r(s,a) + \gamma \cdot \sum_{s' \in \mathcal{S}_{s,a}} \textbf{p} \paren{s' \sachant s,a} \sup_{(d^+) \in \mathcal{D}_{\infty}} V^{d^+}(s')} \\
\nonumber & = & (\mathcal{B}^* \sup_{(d^+) \in \mathcal{D}_{\infty}} V^{d^+})(s) = (\mathcal{B}^* \sup_{(d) \in \mathcal{D}_{\infty}} V^{d})(s) .
\end{eqnarray}
The equality \ref{justif_bellopt} is justified
just like at the line \ref{munos} of the proof of Theorem \ref{thm_mdp_finiteH}.

Thus $V^*= \sup_{d \in \mathcal{D}_{\infty}} V^d$ 
as stated by the Fixed-Point Theorem.
Indeed, as $\mathbb{B}^*$ is a contracting operator (see Property \ref{property_DPcontract}), 
the solution $V^*$ of the Dynamic Programming equation \ref{dynamic_programming_equation} is unique. 

First, as $d^*:  s \mapsto a^* \in \operatorname*{argmax}_{a \in A}(\mathcal{B}^a V^*)(s)$,
\[ \mathcal{B}^{d^*} V^{*} = \max_{a \in \mathcal{A}} (\mathcal{B}^a V^*)(s) = \mathcal{B}^* V^*\]
The function $V^*$ is then a fixed-point of the contracting operator $\mathcal{B}^{d^*}$
(see Property \ref{bellman_contracting}).
As noted earlier, $V^{d^*}$ is a fixed-point of $\mathcal{B}^{d^*}$ too, 
and thus $V^{d^*} = V^* = \max_{(d) \in \mathcal{D}_{\infty}} V^d$. 
It means that $d^*$ is an optimal strategy.
It is thus shown that
it is sufficient to look for 
stationary strategies
because at least one of them,
$d^*$, is optimal.
\end{proof}







\subsection{Proof of Theorem \ref{first_convMDP}}
\label{first_convMDP_RETURN}
\begin{proof}
First,
\[ \normsup{V^N -V^*} = \normsup{ \mathcal{B}^* V^{N-1} - \mathcal{B}^* V^*} \leqslant \gamma \cdot \normsup{ V^{N-1} - V^* } \leqslant \gamma \cdot \paren{ \normsup{V^{N-1} - V^N} + \normsup{V^N - V^*} } \] 
and then $\normsup{V^N -V^*} \leqslant \frac{\gamma}{1-\gamma} \cdot \normsup{ V^{N-1} - V^{N} }$.
Moreover, 
\[ \normsup{ V^{N-1} - V^{N} } = \normsup{ (\mathcal{B}^*)^{N-1} V^{0} - (\mathcal{B}^*)^{N-1} V^{1} } \leqslant \gamma^{N-1} \cdot \normsup{ V^{0} - V^{1} }.\]
Finally,
\begin{equation*} 
\normsup{V^N -V^*} \leqslant \frac{\gamma^N}{1-\gamma} \cdot \normsup{ V^{0} - V^{1} }. 
\end{equation*}
\end{proof}








\subsection{Proof of Theorem \ref{conv_MDP2}}
\label{conv_MDP2_RETURN}
\begin{proof}
The Bellman equation for the strategy $(d)$,
is $V^d = \mathcal{B}^d V^d$, 
and the last iteration of the algorithm is 
$V^{N+1} = \mathcal{B}^* V^{N} = \mathcal{B}^d V^{N}$ 
(as $(d^*)$ is greedy with respect to $V_N$,
we consider $V_{N+1}$ even if it is not actually computed). 
Thanks to these two equalities,
it is possible to write
\begin{eqnarray}
\nonumber \normsup{V^d - V^{N+1}} = \normsup{\mathcal{B}^d V^d - \mathcal{B}^d V^{N}} & \leqslant & \gamma \cdot \normsup{V^d - V^{N}} \\
\nonumber & \leqslant & \gamma \cdot (\normsup{V^d - V^{N+1}} + \normsup{V^{N+1} - V^{N}}) 
\end{eqnarray}
\begin{equation}
\Rightarrow  \normsup{V^d - V^{N+1}} \leqslant \frac{\gamma}{1 - \gamma} \cdot \normsup{V^{N+1} -V^{N}} \leqslant \frac{\gamma^{N+1}}{1 - \gamma} \cdot \normsup{V^1 -V^{0}}.
\label{conv_mdp_2} 
\end{equation}

Finally, thanks to results (\ref{conv_mdp_1}) and (\ref{conv_mdp_2}), 
we get the control of the strategy error:
\[ \normsup{ V^d - V^* } \leqslant \normsup{ V^d - V^N } + \normsup{ V^N - V^* } \leqslant \frac{2 \cdot \gamma^N}{1-\gamma} \normsup{ V^1 - V^0 }. \]
\end{proof}










\subsection{Proof of theorem \ref{belief_process_recursif}}
\label{belief_process_recursif_RETURN}
\begin{proof}
If $i_{t+1}$ is the current information, $\forall s' \in \mathcal{S}$,
\begin{align}
\nonumber & b_{t+1}(s') \\
\nonumber	 		&:= \mathbb{P} \paren{ S_{t+1} = s' \sachant I_{t+1} = i_{t+1}} \\
\label{note_defcond}		&= \frac{ \mathbb{P} \paren{ S_{t+1} = s', O_{t+1}=o_{t+1} \sachant I_t=i_t,a_{t} }}{ 
\mathbb{P} \paren{ O_{t+1} = o_{t+1} \sachant I_t=i_t,a_{t}  } } \\
\nonumber		&= \frac{ \displaystyle \sum_{s \in \mathcal{S}} \mathbb{P} \paren{ S_{t+1} = s', O_{t+1}=o_{t+1} \sachant S_{t}=s,I_t=i_t,a_{t} } \cdot \mathbb{P} \paren{ S_t = s \sachant I_t=i_t,a_{t} }}{ 
\displaystyle \sum_{\tilde{s} \in \mathcal{S}} \mathbb{P} \paren{ O_{t+1} = o_{t+1} \sachant S_t = \tilde{s}, I_t=i_t,a_{t}  } \cdot \mathbb{P} \paren{ S_t = \tilde{s} \sachant I_t=i_t,a_{t} } } \\
\nonumber		&= \frac{ \displaystyle \sum_{s \in \mathcal{S}}  \mathbb{P} \paren{ O_{t+1} = o_{t+1} \sachant \hspace{-0.08cm} S_{t+1}=s', S_{t}=s,I_t=i_t,a_{t} } \cdot \mathbb{P} \paren{ S_{t+1} = s' \sachant \hspace{-0.08cm} S_{t}=s,I_t=i_t,a_{t} } \cdot b_t(s) }{ 
\displaystyle \sum_{\tilde{s} \in \mathcal{S}} \sum_{\tilde{s}' \in \mathcal{S}}  \mathbb{P} \paren{ O_{t+1} = o_{t+1}, S_{t+1}=\tilde{s}' \sachant S_t=\tilde{s}, I_t=i_t,a_{t}  } \cdot b_t(\tilde{s})  } \\
\nonumber	&= \frac{ \displaystyle \sum_{s \in \mathcal{S}}  \textbf{p} \paren{ o_{t+1} \sachant s', a_{t} } \cdot \textbf{p} \paren{ s' \sachant s,a_t } \cdot b_t(s) }{ 
\displaystyle \sum_{\tilde{s} \in \mathcal{S}} \sum_{\tilde{s}' \in \mathcal{S}} \textbf{p} \paren{ o_{t+1} \sachant \tilde{s}', a_{t} } \cdot \textbf{p} \paren{ \tilde{s}' \sachant \tilde{s},a_t } \cdot b_t(\tilde{s})  } := u(b_t,a_{t},o_{t+1})(s').
\end{align}
where line \ref{note_defcond} is simply given by the fact that, for $A$, $B$, $C$ subsets of $\Omega$, 
$\mathbb{P} \paren{ A \sachant B \cap C } = \frac{\mathbb{P} \paren{ A \cap B \sachant C }}{\mathbb{P} \paren{  B \sachant C }}$.
\end{proof}








\subsection{Proof of Theorem \ref{rewriting_probPOMDP}}
\label{rewriting_probPOMDP_RETURN}
\begin{proof}
We use here the following notations: 
$\widehat{A}_{t} = \set{ A_0, \ldots, A_{t} }$,
$\widehat{a}_{t} = \set{ a_0, \ldots, a_{t} }$,
$\widehat{O}_t = \set{ O_1, \ldots, O_t }$,
and $\widehat{o}_t = \set{ o_1, \ldots, o_t }$.
The belief at time step $t$ can be written
as a function of $\widehat{o}_t$, $\widehat{a}_{t-1}$ and $b_0$:
the belief $b_t$ is then denoted by $b^{i_t}_{b_0} = b^{\widehat{o}_t,\widehat{a}_{t-1}}_{b_0}$, and
\begin{equation}
b^{i_t}_{b_0}(s) = b^{\widehat{o}_t,\widehat{a}_{t-1}}_{b_0}(s) = \dfrac{ \mathbb{P} \paren{ S_t = s, \widehat{O}_t = \widehat{o}_t \sachant \widehat{A}_{t-1} = \widehat{a}_{t-1} }  }{ \mathbb{P} \paren{ \widehat{O}_t = \widehat{o}_t \sachant \widehat{A}_{t-1} = \widehat{a}_{t-1} }  } = u \bigg( u\Big( \ldots u \paren{ b_0,a_0,o_1}, \ldots \Big), a_{H-1}, o_H \bigg)(s).
\label{big_bayes}
\end{equation}
The following notation will be useful for the follow-up:
\[ \textbf{p} \paren{ \widehat{o}_t, s_t \sachant \widehat{a}_{t-1}, s_0 } = 
\textbf{p} \paren{ o_t \sachant s_t, a_{t-1} } \cdot \displaystyle \sum_{ s_1, \ldots, s_{t-1} } \textbf{p} \paren{ s_t \sachant s_{t-1}, a_{t-1} } \cdot \displaystyle \prod_{i=1}^{t-1} \textbf{p} \paren{ o_i \sachant s_i, a_{i-1} } \cdot \textbf{p} \paren{ s_i \sachant s_{i-1}, a_{i-1} }.\]
The numerator of the fraction (\ref{big_bayes})
may be written 
\[ \mathbb{P} \paren{ S_t = s, \widehat{O}_t = \widehat{o}_t \sachant \widehat{A}_{t-1} = \widehat{a}_{t-1} } = \displaystyle \sum_{s_0 \in \mathcal{S}} \textbf{p} \paren{ \widehat{o}_t, s_t \sachant \widehat{a}_{t-1}, s_0 } \cdot b_0(s_0). \]
As well, the denominator of the fraction (\ref{big_bayes}) is
\[\mathbb{P} \paren{ \widehat{O}_t = \widehat{o}_t \sachant \widehat{A}_{t-1} = \widehat{a}_{t-1} } = \displaystyle \sum_{s_t \in \mathcal{S}} \sum_{s_0 \in \mathcal{S}}  \textbf{p} \paren{ \widehat{o}_t, s_t \sachant \widehat{a}_{t-1}, s_0 } \cdot b_0(s_0). \]
Finally, the probability of the system state at time step $t$ conditioned on the random sequence of actions is
\begin{align*} 
\mathbb{P} \paren{ S_t = s \sachant \widehat{A}_{t-1} = \widehat{a}_{t-1} } & = \sum_{\substack{ (s_0, \ldots, s_{t-1}) \\ \in \mathcal{S}^t }} \prod_{i=1}^{t} \textbf{p} \paren{ s_i \sachant s_{i-1}, a_{i-1} } \cdot b_0(s_0) \\  
& = \displaystyle \sum_{ \widehat{o}_t} \sum_{ s_0 \in \mathcal{S}}  \textbf{p} \paren{ \widehat{o}_t, s_t \sachant \widehat{a}_{t-1}, s_0 } \cdot b_0(s_0).
\end{align*}
Then, the expectation of the reward conditioned on the random sequence of actions is
\begin{align*}
\mathbb{E} \croch{ r(S_t,A_t) \sachant \widehat{A}_{t} = \widehat{a}_{t} } 
&= \sum_{s \in \mathcal{S}} r(s,a_t) \cdot \mathbb{P} \paren{ S_t = s \sachant \widehat{A}_{t-1} = \widehat{a}_{t-1} } \\
&= \sum_{s \in \mathcal{S}} r(s,a_t) \cdot \sum_{s_0,s_1, \ldots, s_{t-1}} \prod_{i=1}^{t} \textbf{p} \paren{ s_i \sachant s_{i-1}, a_{i-1} } \cdot b_0(s_0) \\
&= \sum_{s \in \mathcal{S}} r(s,a_t) \cdot \sum_{ \widehat{o}_t} \sum_{ s_0 \in \mathcal{S}}  \textbf{p} \paren{ \widehat{o}_t, s \sachant \widehat{a}_{t-1}, s_0 } \cdot b_0(s_0)\\
&= \sum_{ \widehat{o}_t} \sum_{(s_0,s) \in \mathcal{S}^2} \textbf{p} \paren{ \widehat{o}_t, s \sachant \widehat{a}_{t-1}, s_0 } \cdot b_0(s_0) \cdot r(s,a_t).
\end{align*}

Next, multiplying each term of the sum over observations $\widehat{o}_t$ by
\[ 1 =  \dfrac{ \sum_{(s^{(1)}, s'^{(1)}) \in \mathcal{S}^2} \textbf{p} \paren{ \widehat{o}_t, s'^{(1)} \sachant \widehat{a}_{t-1}, s^{(1)} } \cdot b_0(s^{(1)}) }{ \sum_{(s^{(2)}, s'^{(2)}) \in \mathcal{S}^2}\textbf{p} \paren{ \widehat{o}_t, s'^{(2)} \sachant \widehat{a}_{t-1}, s^{(2)} } \cdot b_0(s^{(2)}) },\]
we get
\begin{align}
\nonumber \sum_{ \widehat{o}_t} &	\dfrac{ \sum_{(s^{(1)}, s'^{(1)}) \in \mathcal{S}^2} \textbf{p} \paren{ \widehat{o}_t, s'^{(1)} \sachant \widehat{a}_{t-1}, s^{(1)} } \cdot b_0(s^{(1)}) }{ \sum_{(s^{(2)}, s'^{(2)}) \in \mathcal{S}^2}\textbf{p} \paren{ \widehat{o}_t, s'^{(2)} \sachant \widehat{a}_{t-1}, s^{(2)} } \cdot b_0(s^{(2)}) } \cdot \sum_{(s_0,s) \in \mathcal{S}^2} \textbf{p} \paren{ \widehat{o}_t, s \sachant \widehat{a}_{t-1}, s_0 } \cdot b_0(s_0) \cdot r(s,a_t) \\
\nonumber =& \sum_{ \widehat{o}_t} \sum_{(s^{(1)}, s'^{(1)})} \textbf{p} \paren{ \widehat{o}_t, s'^{(1)} \sachant \widehat{a}_{t-1}, s^{(1)} } \cdot b_0(s^{(1)}) \cdot \sum_{s \in \mathcal{S}} \dfrac{ \displaystyle \sum_{s_0 \in \mathcal{S}} \textbf{p} \paren{ \widehat{o}_t, s \sachant \widehat{a}_{t-1}, s_0 } \cdot b_0(s_0) }{ \displaystyle \sum_{(s^{(2)}, s'^{(2)})}\textbf{p} \paren{ \widehat{o}_t, s'^{(2)} \sachant \widehat{a}_{t-1}, s^{(2)} } \cdot b_0(s^{(2)}) } \cdot r(s,a_t) \\
\nonumber =& \sum_{ \widehat{o}_t} \sum_{(s^{(1)}, s'^{(1)})} \textbf{p} \paren{ \widehat{o}_t, s'^{(1)} \sachant \widehat{a}_{t-1}, s^{(1)} } \cdot b_0(s^{(1)}) \cdot \sum_{s \in \mathcal{S}} \dfrac{ \displaystyle \sum_{s_0 \in \mathcal{S}} \textbf{p} \paren{ \widehat{o}_t, s \sachant \widehat{a}_{t-1}, s_0 } \cdot b_0(s_0) }{ \displaystyle \sum_{(s^{(2)}, s'^{(2)})}\textbf{p} \paren{ \widehat{o}_t, s'^{(2)} \sachant \widehat{a}_{t-1}, s^{(2)} } \cdot b_0(s^{(2)}) } \cdot r(s,a_t) \\
\nonumber =& \sum_{ \widehat{o}_t} \sum_{(s^{(1)}, s'^{(1)})} \textbf{p} \paren{ \widehat{o}_t, s'^{(1)} \sachant \widehat{a}_{t-1}, s^{(1)} } \cdot b_0(s^{(1)}) \cdot \sum_{s \in \mathcal{S}} b^{\widehat{o}_t,\widehat{a}_{t-1}}_{b_0} (s) \cdot r(s,a_t)\\
\nonumber =& \sum_{ \widehat{o}_t} \mathbb{P} \paren{ \widehat{O}_t = \widehat{o}_t \sachant \widehat{A}_{t-1} = \widehat{a}_{t-1} } \cdot \sum_{s \in \mathcal{S}} b^{\widehat{o}_t,\widehat{a}_{t-1}}_{b_0} (s) \cdot r(s,a_t)\\
\nonumber =& \mathbb{E} \croch{ \sum_{s \in \mathcal{S}} B_t(s) \cdot r(s,A_t) \sachant \widehat{A}_{t} = \widehat{a}_{t} }\\
\nonumber =& \mathbb{E} \croch{ r(B_t,A_t) \sachant \widehat{A}_{t} = \widehat{a}_{t} }
\end{align}
where $r(b,a) = \sum_{s \in \mathcal{S}} r(s,a) \cdot b(s)$.
\end{proof}










\subsection{Proof of Theorem \ref{belief_markovien}}
\label{belief_markovien_RETURN}
\begin{proof}
First, $\forall a_t \in \mathcal{A}, \forall o' \in \mathcal{O}$,
\begin{align*}
\mathbb{P} \paren{ O_{t+1}=o' \sachant I_t = i_t, a_t } &= \sum_{s' \in \mathcal{S}} \mathbb{P} \paren{ O_{t+1} = o', S_{t+1} = s' \sachant I_t = i_t, a_t } \\
&= \sum_{(s,s') \in \mathcal{S}^2} \textbf{p} \paren{ o' \sachant s', a_t } \cdot \textbf{p} \paren{ s' \sachant s, a_t } \cdot \mathbb{P} \paren{ S_t=s \sachant I_t=i_t } \\
&= \sum_{(s,s') \in \mathcal{S}^2} \textbf{p} \paren{ o' \sachant s', a_t } \cdot \textbf{p} \paren{ s' \sachant s, a_t } \cdot b_t(s).
\end{align*}
where $b_t = b_{b_0}^{i_t}$ \textit{i.e.} $b_t$ is the belief obtained starting from $b_0$ 
and computed with information $i_t$.
For the sake of readability, 
the result $\sum_{(s,s') \in \mathcal{S}^2} \textbf{p} \paren{ o' \sachant s', a_t } \cdot \textbf{p} \paren{ s' \sachant s, a_t } \cdot b_t(s)$
is denoted by $\textbf{p} \paren{ o' \sachant b_t, a_t }$.
Then, 
\begin{align}
\nonumber \mathbb{P} \paren{ B_{t+1} = b' \sachant I_t=i_t, a_t } &= \mathbb{E} \croch{ \mathds{1}_{\set{ B_{t+1} = b' }} \sachant I_t = i_t, a_t }\\
\nonumber &= \sum_{o' \in \mathcal{O}} \mathbb{P} \paren{ O_{t+1} = o' \sachant I_t = i_t, a_t } \cdot \mathds{1}_{ \set{ u \paren{ b_{b_0}^{i_t}, a_t, o'} = b'} }\\
\nonumber &= \sum_{o' \in \mathcal{O}} \sum_{(s,s') \in \mathcal{S}^2} \textbf{p} \paren{ o' \sachant s', a_t } \cdot \textbf{p} \paren{ s' \sachant s, a_t } \cdot b_t(s) \cdot \mathds{1}_{ \set{ u \paren{ b_t, a_t, o'} = b'} } \\
\label{result_belief_markov} &= \sum_{o' \in \mathcal{O}} \textbf{p} \paren{ o' \sachant b_t, a_t } \cdot \mathds{1}_{ \set{ u \paren{ b_t, a_t, o'} = b'} },
\end{align}
and thus $\mathbb{P} \paren{ B_{t+1} = b' \sachant I_t, a_t }$ is $b^{I_t}_{b_0}$-measurable, \textit{i.e.} $B_t$-measurable.
Indeed, it is shown that $\mathbb{P} \paren{ B_{t+1} = b' \sachant I_t=i_t, a_t }$ is a measurable function of $b_t = b_{b_0}^{i_t}$
when $a_t \in \mathcal{A}$ is fixed.

In order to show the equality (\ref{belief_markov}) asserting that the belief process is a Markov process, 
%\end{equation}
it is sufficient to show the following equation (see Definition \ref{def_conditional_expectation}): 
$\forall (b,b') \in \paren{\mathbb{P}^{\mathcal{S}}_{b_0}}^2$: 
\begin{equation} 
\label{belief_markov_second}
\int_{ \set{B_t = b} } \mathbb{E} \croch{ \mathds{1}_{\set{ B_{t+1} = b' }} \sachant I_t, a_t }  d \mathbb{P} = \int_{ \set{B_t = b} } \mathds{1}_{\set{ B_{t+1} = b'}}  d \mathbb{P}. 
\end{equation}
Indeed, as $\mathbb{P} \paren{ B_{t+1} = b' \sachant I_t, a_t  } = \mathbb{E} \croch{ \mathds{1}_{\set{ B_{t+1} = b' }} \sachant I_t, a_t } $ is $B_t$-measurable,
it remains to show the equality (\ref{belief_markov_second}) 
to prove that $\mathbb{E} \croch{ \mathds{1}_{\set{ B_{t+1} = b' }} \sachant I_t, a_t } = \mathbb{E} \croch{ \mathds{1}_{\set{ B_{t+1} = b' }} \sachant B_t, a_t }$ $\mathbb{P}$-amolst surely, \textit{i.e.} to show equality (\ref{belief_markov}).

On the one hand, the left part of the equation (\ref{belief_markov_second}) is
\begin{align} 
\nonumber\int_{ \set{B_t = b} }& \sum_{o' \in \mathcal{O}} \textbf{p} \paren{ o' \sachant b_t, a_t } \cdot \mathds{1}_{ \set{ u \paren{ b_t, a_t, o'} = b'} } d \mathbb{P} \\
\label{belief_markov_l} &= \int_{ \Omega } \mathds{1}_{\set{B_t = b}} d \mathbb{P} \cdot \sum_{o' \in \mathcal{O}} \textbf{p} \paren{ o' \sachant b, a_t } \cdot \mathds{1}_{ \set{ u \paren{ b, a_t, o'} = b'} }.
\end{align}
thanks to the equality (\ref{result_belief_markov}).

On the other hand, the right part of equation \ref{belief_markov_second} is
\begin{align} 
\nonumber \int_{ \set{B_t = b} } \mathds{1}_{\set{ B_{t+1} = b'}}  d \mathbb{P} &= \int_{ \Omega} \mathds{1}_{\set{B_t = b} } \cdot \mathds{1}_{\set{ u(b,a_t,O_{t+1}) = b'}}  d \mathbb{P}  \\
\nonumber &= \mathbb{E} \croch{ \mathds{1}_{\set{B_t = b} } \cdot \mathds{1}_{\set{ u(b,a_t,O_{t+1}) = b'}} }\\
\label{explain_fromdef} &= \mathbb{E} \Bigg[ \mathbb{E} \croch{ \mathds{1}_{\set{B_t = b} } \cdot \mathds{1}_{\set{ u(b,a_t,O_{t+1}) = b'}} \sachant I_t, a_t } \Bigg]\\
\label{explain_btItmeas} &= \mathbb{E} \Bigg[ \mathds{1}_{\set{B_t = b} } \cdot \mathbb{E} \croch{  \mathds{1}_{\set{ u(b,a_t,O_{t+1}) = b'}} \sachant I_t, a_t } \Bigg]\\
\label{explain_lastline} &= \mathbb{E} \Bigg[ \mathds{1}_{\set{B_t = b} } \cdot \sum_{o' \in \mathcal{O}} \textbf{p} \paren{ o' \sachant b, a_t } \cdot \mathds{1}_{ \set{ u \paren{ b, a_t, o'} = b'} } \Bigg],
\end{align}
which is also equal to result (\ref{belief_markov_l}).
Line (\ref{explain_fromdef}) comes from Definition \ref{def_conditional_expectation}.
Line (\ref{explain_btItmeas}) comes from Property \ref{prop_def_condexpec} 
and the fact that $B_t$ (and thus $\mathds{1}_{\set{B_t = b}}$) is $\sigma(I_t)$-measurable.
The last line (\ref{explain_lastline}) is given by the result (\ref{result_belief_markov}).

The belief process $(B_t)_{t \geqslant 0}$ is thus a Markov process.
\end{proof}











\subsection{Proof of Theorem \ref{PWLC_theorem}}
\label{PWLC_theorem_RETURN}
\begin{proof}
Let $V: \mathbb{P}^{\mathcal{S}}_{b_0} \rightarrow \mathbb{R}$ be a PWLC function. 
Then, $\exists \Gamma \subset \mathbb{R}^{\mathcal{S}}$, $\# \Gamma < + \infty$ such that 
\[ V(b)=\max_{\alpha \in \Gamma} \set{ \sum_{s \in \mathcal{S}} b(s) \cdot \alpha(s) } = \max_{\alpha \in \Gamma} \langle \alpha, b \rangle_{\mathbb{R}^{\mathcal{S}}}  . \] 

For $b \in \mathbb{P}^{\mathcal{S}}_{b_0}$, the Dynamic Programming Equation is
\begin{eqnarray*}
\paren{\mathcal{B}^* V}(b) & = & \max_{a \in \mathcal{A}} \set{ \sum_{s \in \mathcal{S}} r(s,a) \cdot b(s) + \gamma \cdot \sum_{o' \in \mathcal{O}} \textbf{p} \paren{o' \sachant b , a } \cdot V \big( u(b,a,o') \big) } \\
& = & \max_{a \in \mathcal{A}} \set{ \sum_{s \in \mathcal{S}} r(s,a) \cdot b(s) + \gamma \cdot \sum_{o' \in \mathcal{O}} \textbf{p} \paren{o' \sachant b , a } \cdot \max_{\alpha \in \Gamma} \bigg\{ \sum_{s' \in \mathcal{S}} u(b,a,o')(s') \cdot \alpha(s') \bigg\} } \\
& = & \max_{a \in \mathcal{A}} \set{ \sum_{s \in \mathcal{S}} r(s,a) \cdot b(s) +  \max_{(\alpha_{o})_{o \in \mathcal{O}} \in \Gamma^\mathcal{O}} \bigg\{ \gamma \cdot \sum_{o' \in \mathcal{O}} \textbf{p} \paren{o' \sachant b , a } \cdot \sum_{s' \in \mathcal{S}} u(b,a,o')(s') \cdot \alpha_{o'}(s') \bigg\} } \\
& = & \max_{a \in \mathcal{A}} \max_{(\alpha_{o})_{o \in \mathcal{O}} \in \Gamma^\mathcal{O}} \set{ \sum_{s \in \mathcal{S}} r(s,a) \cdot b(s) + \bigg\{ \gamma \cdot \sum_{o' \in \mathcal{O}} \textbf{p} \paren{o' \sachant b , a } \cdot \sum_{s' \in \mathcal{S}} u(b,a,o')(s') \cdot \alpha_{o'}(s') \bigg\} }
\end{eqnarray*}
where $(\alpha_{o})_{o \in \mathcal{O}}$ is an element of $\Gamma^\mathcal{O}$
\textit{i.e.} is a set of vectors from $\Gamma \subset \mathbb{R}^{\mathcal{S}}$:
each vector $\alpha_{o} \in \Gamma$ is indexed by an observation $o \in \mathcal{O}$.

Thereafter, given a belief $b \in \tilde{\mathcal{S}}$
and an action $a \in \mathcal{A}$, 
we use the notation $\textbf{p} \paren{ s' \sachant b,a } = \sum_{s \in \mathcal{S}} \textbf{p} \paren{ s' \sachant s, a } \cdot b(s).$
Then, the belief update (\ref{belief_update}) becomes
$\forall s' \in \mathcal{S}$,
\[ u \paren{b,a,o'}(s') =  \dfrac{ \textbf{p} \paren{o' \sachant s',a} \cdot \textbf{p} \paren{ s' \sachant b ,a } }{ \textbf{p} \paren{ o' \sachant b,a}}.  \]
The result applying $\mathcal{B}^*$ to $V$ is then
\begin{eqnarray*}
\paren{\mathcal{B}^* V}(b) & = & \max_{a \in \mathcal{A}} \max_{(\alpha_{o'}) \in \Gamma^{\mathcal{O}} } \sum_{s \in \mathcal{S}} r(s,a) \cdot b(s) + \gamma \cdot \sum_{o' \in \mathcal{O}} \textbf{p} \paren{o' \sachant b, a} \cdot \sum_{s' \in \mathcal{S}} \dfrac{ \textbf{p} \paren{o' \sachant s',a} \cdot \textbf{p} \paren{ s' \sachant b ,a } }{ \textbf{p} \paren{ o' \sachant b,a}} \cdot \alpha_{o'}(s') \\
& = & \max_{a \in \mathcal{A},(\alpha_{o'}) \in \Gamma^{\mathcal{O}}}  \sum_{s \in \mathcal{S}} r(s,a) \cdot b(s) + \gamma \cdot \sum_{o' \in \mathcal{O}} \sum_{s' \in \mathcal{S}} \textbf{p} \paren{o' \sachant s',a} \cdot \sum_{s \in \mathcal{S}} \textbf{p} \paren{ s' \sachant s ,a  } \cdot b(s) \cdot \alpha_{o'}(s') \\
& = & \max_{a \in \mathcal{A},(\alpha_{o'}) \in \Gamma^{\mathcal{O}}}  \sum_{s \in \mathcal{S}} \paren{ r(s,a) + \gamma \cdot \sum_{o' \in \mathcal{O}} \sum_{s' \in \mathcal{S}} \textbf{p} \paren{o' \sachant s',a} \cdot \textbf{p} \paren{ s' \sachant s ,a  } \cdot \alpha_{o'}(s') } \cdot b(s) \\
& = & \max_{\alpha' \in \Gamma'}  \sum_{s \in \mathcal{S}} \alpha'(s) \cdot b(s) = \max_{\alpha' \in \Gamma'} \langle \alpha', b \rangle_{\mathbb{R}^{\mathcal{S}}}.
\end{eqnarray*}
where
\[\Gamma' = \set{ \alpha'(s) = r(s,a) + \gamma \cdot \displaystyle \sum_{o' \in \mathcal{O}} \sum_{s' \in \mathcal{S}} \textbf{p} \paren{o' \sachant s',a} \cdot \textbf{p} \paren{ s' \sachant s ,a  } \cdot \alpha_{o'}(s') \sachant a \in \mathcal{A}, \mbox{ and } \forall o' \in \mathcal{O}, \alpha_{o'} \in \Gamma }\]
which is finite, and which size is $\# \mathcal{A} \cdot (\# \Gamma)^{ \# \mathcal{O} }$.

The function $b \mapsto \mathcal{B}^* V(b)$ is thus PWLC 
and elements of $\Gamma'$ are called $\alpha$-vectors. 
\end{proof}














\subsection{Proof of Theorem \ref{theorem_NIequivalence}}
\label{theorem_NIequivalence_RETURN}
\begin{proof}
First, note that
\begin{align*}
\forall (x,y) \in \mathcal{X} \times \mathcal{Y}, \hspace{0.5cm} & \pi(x,y) < \min \set{ \pi(x), \pi(y) } \\
 &\Updownarrow \\
\forall (x,y) \in \mathcal{X} \times \mathcal{Y}, \hspace{0.5cm} & \pi(x,y) < \max_{x'} \pi(x',y) = \pi(y) \mbox{ and } \pi(x,y) < \max_{y'} \pi(x,y') = \pi(x)\\
& \Updownarrow \\
\forall (x,y) \in \mathcal{X} \times \mathcal{Y}, \hspace{0.5cm} & \pi\paren{x \sachant y} = \pi(x,y) < \pi(x) \mbox{ and } \pi\paren{y \sachant x} = \pi(x,y) < \pi(y)
\end{align*}
Thus, as $\pi(x,y) \leqslant \min \set{ \pi(x), \pi(y) }$ is always true,
$\forall (x,y) \in \mathcal{X} \times \mathcal{Y}$, $\pi(x,y) = \min \set{ \pi(x), \pi(y) }$
$\Leftrightarrow$ $\forall (x,y) \in \mathcal{X} \times \mathcal{Y}$, $\pi\paren{x \sachant y} \geqslant \pi(x) \mbox{ or } \pi\paren{y \sachant x} \geqslant \pi(y)$.
\end{proof}







\subsection{Proof of Theorem \ref{MS2_ignor}}
\label{MS2_ignor_RETURN}
\begin{proof}
Let us suppose that $X$ and $Y$ are MS-independent: in terms of possibility distributions,
it can be written $\forall x \in \mathcal{X}$, $\forall y \in \mathcal{Y}$, 
$\pi(x)=\pi \paren{ x \sachant y}$ and $\pi(y) = \pi \paren{ y \sachant x }$.
Suppose also that $\exists x_0 \in \mathcal{X}$ such that $\pi(x_0)<1$ 
\textit{i.e.} variable $X$ is not fully unknown.
Thus, using the qualitative possibilistic conditioning (Definition \ref{def_cond}),
we know that, since $\pi \paren{ x_0 \sachant y} = \pi(x_0) <1$, 
$\forall y \in \mathcal{Y}$, $\pi \paren{ x_0 \sachant y} = \pi \paren{x_0, y}$.
Thus, $\pi \paren{x_0, y} < 1$.
Using also the conditioning of Definition \ref{def_cond}, 
we get that $\forall y \in \mathcal{Y}$, $\pi \paren{ y \sachant x_0 }=1$,
since the equality $\pi(x_0) = \pi \paren{x_0, y}$ is the condition which leads to the possibility degree $1$.
As $\forall x \in \mathcal{X}$, $\forall y \in \mathcal{Y}$, $\pi(y) = \pi \paren{ y \sachant x }$,  $\forall y \in \mathcal{Y}$, $\pi \paren{ y  }=1$.
As a conclusion, if $X$ is not fully unknown, $Y$ is is fully unknown: 
thus, if $Y$ is not fully unknown, $X$ is fully unknown.
\end{proof}




\subsection{Proof of Property \ref{property_minmax}}
\label{property_minmax_RETURN}
\begin{proof}
First, if $f^* = \min_{\omega \in \Omega} f(\omega) $,
then $\forall \omega \in \Omega$, $f^* \leqslant f(\omega)$.
Thus $\forall \omega \in \Omega$, $1-f^* \geqslant 1-f(\omega)$, \textit{i.e.}
$1-f^* = 1-\min_{\omega \in \Omega} f(\omega) = \max_{\omega \in \Omega} \set{ 1-f(\omega) }$:
the equation (\ref{equationmaxmin1}) is proved.
The equation (\ref{equationmaxmin2}) can be shown using the previous one: 
setting $f = 1-g$, we get $\forall g: \Omega \rightarrow \mathcal{L}$ 
$\max_{\omega \in \Omega} g(\omega) = 1 - \min_{\omega \in \Omega} \set{ 1 - g(\omega) }$
\textit{i.e.} it equality is shown.

Now, the equation (\ref{equationmaxmin3}) is shown 
considering the case where $\lambda \geqslant \max_{\omega \in \Omega} f(\omega)$:
on the one hand, in this case, both parts of the equality are equal to $\max_{\omega \in \Omega} f(\omega) $
since $\forall \omega \in \Omega$, $\lambda \geqslant f(\omega)$.
On the other hand, when $\lambda \leqslant \max_{\omega \in \Omega} f(\omega)$, 
right part of equation (\ref{equationmaxmin3}) is equal to $\lambda$. 
As $\forall \omega \in \Omega$, $\min \set{ \lambda, f(\omega) } \leqslant \lambda$,
and for each $\omega$ such that $f(\omega) \geqslant \lambda$, $\min \set{ \lambda, f(\omega) } = \lambda$, 
the maximum is equal to $\lambda$: the left part of the equation (\ref{equationmaxmin3}) is equal to $\lambda$ too. 
The equation (\ref{equationmaxmin4}) can be be proved using the previous equation:
setting $f$ to $1-g$, and $\lambda$ to $1 - \mu$, we get
$\max_{\omega \in \Omega} \min \set{ 1 - g(\omega), 1 - \mu } = \min \set{ \max_{\omega \in \Omega} \set{1 - g(\omega)}, 1 - \mu }$.
And then, using equations (\ref{equationmaxmin1}) and (\ref{equationmaxmin2}), 
$\forall \mu \in \mathcal{L}$, $\forall g:\Omega \rightarrow \mathcal{L}$,
$1 - \min_{\omega \in \Omega} \max \set{ g(\omega), \mu } = 1 - \max \set{ \min_{\omega \in \Omega} g(\omega), \mu }$,
\textit{i.e.} the equation is shown.

The equations (\ref{equationmaxmin5}) and (\ref{equationmaxmin6})
are trivial: with the operator $\min$ (resp. $\max$), 
no matter what is the order of elements and the repetitions,
as they are associative\footnote{ An operator $*$ over $\mathcal{L}$ is associative if $\forall (\lambda_1,\lambda_2,\lambda_3) \in \mathcal{L}^3$, $(\lambda_1 * \lambda_2) * \lambda_3 = \lambda_1 * (\lambda_2 * \lambda_3)$.}.

The inclusions (\ref{equationargmax1}) and (\ref{equationargmax2})
are shown as follows: let $\omega^* \in \operatorname*{argmax}_{\omega \in \Omega} f(\omega)$.
If $f(\omega^*) \leqslant \lambda$, $\forall \omega \in \Omega$, 
$\min \set{ f(\omega), \lambda } = f(\omega)$ and thus
$\operatorname*{argmax}_{\omega \in \Omega} \min \set{ f(\omega), \lambda } = \operatorname*{argmax}_{\omega \in \Omega} f(\omega)$.
Now, if $f(\omega^*) > \lambda$, $\max_{\omega \in \Omega} \min \set{ f(\omega), \lambda } = \lambda$, 
and then $\operatorname*{argmax}_{\omega \in \Omega} \min \set{ f(\omega), \lambda } = \set{ \omega \sachant f(\omega) \geqslant \lambda }$:
thus $\operatorname*{argmax}_{\omega \in \Omega} f(\omega) \subseteq \operatorname*{argmax}_{\omega \in \Omega} \min \set{ f(\omega), \lambda }$,
\textit{i.e.} the inclusion (\ref{equationargmax1}) is shown.
For the inclusion (\ref{equationargmax2}), 
if $f(\omega^*)> \lambda$, then the maximizing elements are in $\omega \set{ \omega \sachant f(\omega) > \lambda }$,
where $f(\omega) = \max \set{ f(\omega), \lambda }$.
Thus maximizing elements are the same $\operatorname*{argmax}_{\omega \in \Omega} f(\omega) = \operatorname*{argmax}_{\omega \in \Omega} \max \set{ f(\omega), \lambda }$.
Otherwise, if $f(\omega^*) \leqslant \lambda$, $\forall \omega \in \Omega$,
$\max \set{ f(\omega), \lambda } = \lambda$, then $\operatorname*{argmax}_{\omega \in \Omega} \max \set{ f(\omega), \lambda } = \Omega$,
and thus obviously 
$\operatorname*{argmax}_{\omega \in \Omega} f(\omega) \subseteq \operatorname*{argmax}_{\omega \in \Omega} \max \set{ f(\omega), \lambda }$.

Finally, let us show the equality (\ref{equationmaxmin7}).
Note first that functions $A: \omega \mapsto \max \set{ \min \set{ \lambda, f(\omega) }, g(\omega) }$ 
and $B: \omega \mapsto \min \set{ \lambda, \max \set{ f(\omega), g(\omega) } }$ are equal to
$\omega \mapsto \max \set{ f(\omega), g(\omega) }$ on the set $\Omega_1 = \set{ \omega \sachant \lambda \geqslant \max \set{ f(\omega), g(\omega) }}$.
It is trivial for $B$, and as $\lambda \geqslant f(\omega)$ on this set, the result comes for $A$.
On the set $\Omega_2 = \overline{\Omega_1} = \set{ \omega \sachant \lambda \leqslant \max \set{ f(\omega), g(\omega) } }$,
$B$ is of course equal to $\lambda$, and if $f(\omega) \geqslant \lambda$, $A(\omega) = \max \set{ \lambda, g(\omega) } \geqslant \lambda$,
otherwise if $f(\omega) \leqslant \lambda$, $A(\omega) = \max \set{ f(\omega), g(\omega) } \geqslant \lambda$ (on the set $\Omega_2$).
$A$ and $B$ are thus smaller on $\Omega_1$.
Indeed, they are both equal to $\omega \mapsto \max \set{ f(\omega), g(\omega) } \leqslant \lambda$ on $\Omega_1$,
whereas on $\Omega_2$, $N=\lambda$ and $A \geqslant \lambda$.
Thus, if $\Omega_1 \neq \emptyset$, then the result is shown: the minimum is on $\Omega_1$, where functions are equal.
Otherwise, if $\Omega_1 = \emptyset$, $\forall \omega \in \Omega$, $\lambda \leqslant \max \set{ f(\omega), g(\omega) }$.
As $g(\omega^*)=0$, then $f(\omega^*) \geqslant \lambda$, and $A(\omega^*) = \max \set{ \lambda, g(\omega^*) } = \lambda = B(\omega^*)$.
For the other $\omega \in \Omega$, $A(\omega) = \max \set{ \lambda, g(\omega) } \geqslant \lambda$
or $A(\omega) =\max \set{ f(\omega), g(\omega) } \geqslant \lambda$, thus the minimum of both functions, $\lambda$, is reached with $\omega^*$,
where $A$ and $B$ are equal.
The equality (\ref{equationmaxmin8}) can be proved using the equality (\ref{equationmaxmin7}):
if we set $g=1-h$, $g$ fulfills the condition $\exists \omega^* \in \Omega$ such that $g(\omega^*)=0$.
Setting $\lambda = 1 - \mu$ and $f' = 1 -f$, we get
\[ \min_{\omega \in \Omega} \max \Big\{ \min \set{ 1 -\mu, 1 - f'(\omega) }, 1 - h(\omega)  \Big\} = \min_{\omega \in \Omega} \min \Big\{ 1 -\mu, \max \set{ 1 - f'(\omega) , 1 - h(\omega) }  \Big\}.\]
Using equations (\ref{equationmaxmin1}) and (\ref{equationmaxmin2}) of Property \ref{property_minmax},
we get
\[ 1 - \max_{\omega \in \Omega} \min \Big\{ \max \set{ \mu, f'(\omega) }, h(\omega)  \Big\} = 1 - \max_{\omega \in \Omega} \max \Big\{ \mu, \min \set{ f'(\omega) , h(\omega) }  \Big\}, \]
\textit{i.e.} one minus equation (\ref{equationmaxmin8}).
\end{proof}









\subsection{Proof of the equality of Definition \ref{sugeno_integral}}
\label{sugeno_integral_RETURN}
\begin{proof}
Let us show that (\ref{equation_sugeno1}) is equal to (\ref{equation_sugeno2}).
Note first that, by definition, $f(\omega_i)$ is non-decreasing with $i \in \set{1, \ldots, \# \Omega}$.
Note as well that $\mu$ is monotone, and then $\mu(A_i) = \mu( \set{\omega_i, \ldots, \omega_{\# \Omega}} ) \geqslant \mu(A_{i+1})$
since $A_{i+1} \subset A_i$.
Let $i^*$ be the highest $i \in \set{ 1, \ldots, \# \Omega }$ such that $\mu(A_{i^*}) \geqslant f(\omega_{i^*})$.
As $\mu(A_1) = \mu(\Omega) = 1 \geqslant f(\omega_1)$, $i^*$ exists. 
For each $i \leqslant i^*$, $\min \set{ f(\omega_i), \mu(A_i) } = f(\omega_i)$,
and for each $i > i^*$, $\min \set{ f(\omega_i), \mu(A_i) } = \mu(A_i)$,
thanks to the definition of $i^*$.
As $f(\omega_i)$ is non-decreasing and $\mu(A_i)$ is non-increasing with $i$,
highest values of $\Big( \displaystyle \min \big\{ f(\omega_i), \mu(A_i) \big\} \Big)_{i=1}^{\# \Omega-1}$
are $f(\omega_{i^*})$ and $\mu(A_{i^*+1})$.

If $f(\omega_{i^*}) \leqslant \mu(A_{i^*+1})$, then
 (\ref{equation_sugeno1}) is equal to $\mu(A_{i^*+1})$.
As well, $\max \set{f(\omega_{i^*}), \mu(A_{i^*+1})} = \mu(A_{i^*+1})$.
Using the definition of $i^*$, and as $\mu$ is monotone, 
$f(\omega_{i^*+1}) > \mu \paren{ A_{i^*+1}} \geqslant \mu \paren{ A_{i^*+2} }$.
This implies that $\max \set{f(\omega_{i^*+1}), \mu(A_{i^*+2})} = f(\omega_{i^*+1})$.
As $\mu(A_{i+1})$ is non-increasing with $i$, and $f(\omega_i)$ non-decreasing,
$\mu(A_{i^*+1})$ and $f(\omega_{i^*+1})$ are the lowest values of 
$\Big( \displaystyle \max \big\{ f(\omega_i), \mu(A_{i+1}) \big\} \Big)_{i=1}^{\# \Omega-1}$.
By definition of $i^*$, $f(\omega_{i^*+1})>\mu(A_{i^*+1})$, 
and thus formula (\ref{equation_sugeno2})
is also equal to $\mu(A_{i^*+1})$: (\ref{equation_sugeno1}) and (\ref{equation_sugeno2}) are equal.

If $f(\omega_{i^*}) > \mu(A_{i^*+1})$,
then formula (\ref{equation_sugeno1}) is equal to $f(\omega_{i^*})$,
and $\max \set{ f(\omega_{i^*}), \mu(A_{i^*+1}) } = f(\omega_{i^*})$.
As $f(\omega_i)$ is non-decreasing with $i$, and thanks to the definition of $i^*$,
$f(\omega_{i^*-1}) \leqslant f(\omega_{i^*}) \leqslant \mu(A_{i^*})$
and thus $\max \set{ f(\omega_{i^*-1}), \mu(A_{i^*}) } = \mu(A_{i^*})$.
As previously ($\mu(A_{i+1})$ non-increasing and $f(\omega_i)$ non-decreasing),
$\mu(A_{i^*})$ and $f(\omega_{i^*})$ are the lowest values of 
$\Big( \displaystyle \max \big\{ f(\omega_i), \mu(A_{i+1}) \big\} \Big)_{i=1}^{\# \Omega-1}$.
By definition of $i^*$, $f(\omega_{i^*})\leqslant\mu(A_{i^*})$, 
and thus formula (\ref{equation_sugeno2})
is also equal to $f(\omega_{i^*})$. 

Finaly, formula (\ref{equation_sugeno1}) and (\ref{equation_sugeno2}) are equal.
\end{proof}









\subsection{Proof of Theorem \ref{sugenoPossNec}}
\label{sugenoPossNec_RETURN}
\begin{proof}
First, let us rewrite the Sugeno integral of a function $f:\Omega \rightarrow \mathcal{L}$ 
with respect to a possibility measure $\Pi$, using formula (\ref{equation_sugeno1}): 
\begin{align}
\nonumber \mathbb{S}_{\Pi}[f] &= \max_{i=1}^{\# \Omega} \min \set{ f(\omega_i), \Pi(A_i) } \\
\nonumber &= \max_{i=1}^{\# \Omega} \min \set{  f(\omega_i), \max_{j=i}^{\# \Omega} \pi(\omega_j)  } \\
\label{equation_sugenomax} &= \max_{\substack{ (i,j) \in \set{1,\ldots,\# \Omega}^2  \\ \mbox{ \tiny s.t. } i \leqslant j} } \min \set{  f(\omega_i),  \pi(\omega_j)  }. 
\end{align}
Now, note that $\forall (i,j) \in \set{1,\ldots,\# \Omega}^2$ such that $i<j$,
$\min \set{ f(\omega_i), \pi(\omega_j) } \leqslant \min \set{ f(\omega_j), \pi(\omega_j) } $
since $f(\omega_i)$ is non-decreasing with $i$.
As just shown, such pairs 
have a minimum lower than the minimum of an other pair: 
such pairs can thus be removed 
from the maximum operator 
in the equation (\ref{equation_sugenomax}):
we get then the formula (\ref{equation_sugenoposs}).

As well, using formula (\ref{equation_sugeno2}),
the Sugeno integral of $f$ with respect to a necessity measure $\mathcal{N}$ is
\begin{align}
\nonumber \mathbb{S}_{\mathcal{N}}[f] &= \min_{i=1}^{\# \Omega} \max \set{ f(\omega_i), \mathcal{N}(A_{i+1}) } \\
\nonumber &= \min_{i=1}^{\# \Omega} \max \Big\{ f(\omega_i), 1 - \Pi \big( \set{\omega_1,\ldots,\omega_i} \big) \Big\} \\
\nonumber &= \min_{i=1}^{\# \Omega} \max \set{  f(\omega_i), \min_{j=1}^{i} \set{1 - \pi(\omega_j)}  } \\
\label{equation_sugenomin} &= \min_{\substack{ (i,j) \in \set{1,\ldots,\# \Omega}^2  \\ \mbox{ \tiny s.t. } i \geqslant j} } \min \set{  f(\omega_i), 1 - \pi(\omega_j)  }. 
\end{align}
Note that the terms of $\displaystyle \min_{\substack{ (i,j) \in \set{1,\ldots,\# \Omega}^2  \\ \mbox{ \tiny s.t. } i \geqslant j} }$,
such that $i>j$, \textit{i.e.} $\max \set{ f(\omega_i), 1-\pi(\omega_j) }$,
are greater or equal to $\max \set{ f(\omega_j), 1-\pi(\omega_j) }$
(as $f(\omega_i)$ is non-dreasing with $i$)
and can be removed from $\displaystyle \min_{(i,j)}$:
the equation (\ref{equation_sugenonec}) is then deduced.
\end{proof}





\subsection{Proof of Theorem \ref{DPpiMDP}}
\label{DPpiMDP_RETURN}
\begin{proof}
For the case $i=0$, 
$\overline{U_0^*}(s) = \Psi(s)$, is obvious since no action has to be chosen.
The case $i=1$ consists in applying the formula (\ref{equationmaxmin3}) of Property \ref{property_minmax}.
Following sequence of equalities come from properties (\ref{property_minmax}). 
Let $i$ be in $\set{2, \ldots, H-1}$ and $j=H-i$, 
and $\mathcal{T}_i$ the set of $i$-length system state trajectories $\mathcal{T} = (s_{j+1},\ldots,s_H)$: 
$\forall s_j \in \mathcal{S}$,\\
\\
$\overline{U^*_i}(s_j)$
\vspace{-0.5cm}
\begin{eqnarray}
\nonumber  & = & \displaystyle \max_{(\delta) \in \Delta_i} \max_{\mathcal{T} \in \mathcal{T}_i} \min \set{ \min_{t=j}^{H} \rho_t\Big(s_t,\delta_t(s_t)\Big), \min_{t=j}^{H-1} \pi_t \Big(s_{t+1} \Big\vert s_t , \delta_t(s_t) \Big) } \\
\nonumber & = &  \displaystyle \max_{(\delta) \in \Delta_i} \max_{s_{j+1} \in \mathcal{S}} \max_{\mathcal{T} \in \mathcal{T}_{i-1}} \min \Bigg\{ \min \set{ \rho_{j} \Big( s_{j},\delta_{j}(s_{j}) \Big), \min_{t=j+1}^{H} \rho_{t} \Big(s_t,\delta_t(s_t)\Big)}, \min_{t=j}^{H-1} \pi_t \Big( s_{t+1} \Big\vert s_t , \delta_t(s_t) \Big) \Bigg\} \\
\nonumber & = &  \displaystyle \max_{(\delta) \in \Delta_i} \max_{s_{j+1} \in \mathcal{S}} \min \Bigg\{ \rho_j \Big( s_j,\delta_j(s_j) \Big), \max_{\mathcal{T} \in \mathcal{T}_{i-1}} \min \set{  \min_{t=j+1}^{H} \rho_t \Big(s_t,\delta_t(s_t) \Big), \min_{t=j}^{H-1} \pi_t \Big(s_{t+1} \Big\vert s_t , \delta_t(s_t) \Big) } \Bigg\} \\
\nonumber & = &  \displaystyle \max_{(\delta) \in \Delta_i} \max_{s_{j+1} \in \mathcal{S}} \min \Bigg\{ \rho_j \Big( s_j,\delta_j(s_j) \Big), \min \bigg\{ \pi_j \Big( s_{j+1} \Big\vert s_j, \delta_j(s_j) \Big), \\ 
\nonumber & & \hspace{5cm} \max_{\mathcal{T} \in \mathcal{T}_{i-1}} \min \Big\{  \min_{t=j+1}^{H} \rho_t \Big( s_t,\delta_t(s_t) \Big), \min_{t=j+1}^{H-1} \pi_t \Big( s_{t+1} \Big\vert s_t , \delta_t(s_t) \Big) \Big\} \bigg\} \Bigg\} \\
\nonumber & = &  \displaystyle \max_{\delta_j(s_j) \in \mathcal{A}} \max_{s_{j+1} \in \mathcal{S}} \min \Bigg\{ \rho_j \Big( s_j,\delta_j(s_j) \Big), \min \bigg\{ \pi_j \Big( s_{j+1} \Big\vert s_j, \delta_j(s_j) \Big), \\ 
\nonumber & & \hspace{3.5cm} \max_{(\delta) \in \Delta_{i-1}} \max_{\mathcal{T} \in \mathcal{T}_{i-1}} \min \Big\{  \min_{t=j+1}^{H} \rho_t \Big( s_t,\delta_t(s_t) \Big), \min_{t=j+1}^{H-1} \pi_t \Big( s_{t+1} \Big\vert s_t , \delta_t(s_t) \Big) \Big\} \bigg\} \Bigg\} \\
\nonumber & = &  \displaystyle \max_{a \in \mathcal{A}} \max_{s_{j+1} \in \mathcal{S}} \min \Bigg\{ \rho_j \Big( s_j,a \Big), \min \bigg\{ \pi_j \Big( s_{j+1} \Big\vert s_j, a \Big),\overline{U^*_{i-1}}(s_1) \bigg\} \Bigg\} \\
\nonumber & = &  \displaystyle \max_{a \in \mathcal{A}}  \min \Bigg\{ \rho_j \Big( s_j,a \Big), \max_{s_{j+1} \in \mathcal{S}} \min \bigg\{ \pi_j \Big( s_{j+1} \Big\vert s_j, a \Big),\overline{U^*_{i-1}}(s_1) \bigg\} \Bigg\} .
\end{eqnarray}
where the final preference function $\Psi(s)$ is denoted by $\rho_H(s,a)$ to simplify equations.
This shows that the optimistic value function can be computed using the recursive formula (\ref{equation_recursiveopt}).
The strategy computed with formula (\ref{equation_recursiveoptstrat}) 
is indeed optimal, thanks to the inclusion (\ref{equationmaxmin3}) of Property \ref{property_minmax}.

As well, for the pessimistic criterion,
the case $i=0$ is obvious too, and the case $i=1$ consists in applying the formulae 
(\ref{equationmaxmin1}) and (\ref{equationmaxmin4}) of Property \ref{property_minmax}.
For $i \in \set{2, \ldots, H-1}$, and $j=H-i$, using Property \ref{property_minmax},\\ 
\\
$\underline{U^*_H}(s_j)$ \vspace{-0.5cm}
\begin{eqnarray}
\nonumber  & = & \displaystyle \max_{(\delta) \in \Delta_i} \min_{\mathcal{T} \in \mathcal{T}_i} \max \set{ \min_{t=j}^{H} \rho_t \Big( s_t,\delta_t(s_t) \Big), 1- \min_{t=j}^{H-1} \pi_t \Big( s_{t+1} \Big\vert s_t , \delta_t(s_t) \Big) } \\
\nonumber & = &  \displaystyle \max_{(\delta) \in \Delta_i} \min_{s_{j+1} \in \mathcal{S}} \min_{\mathcal{T} \in \mathcal{T}_{i-1}} \max \Bigg\{ \min \set{ \rho_j \Big( s_j,\delta_j(s_j) \Big), \min_{t=j+1}^{H} \rho_t(s_t,\delta_t(s_t))}, 1- \min_{t=j}^{H-1} \pi_t \Big( s_{t+1} \Big\vert s_t , \delta_t(s_t) \Big) \Bigg\} \\
\label{maxminlambdafg} & = &  \displaystyle \max_{(\delta) \in \Delta_i} \min_{s_{j+1} \in \mathcal{S}} \hspace{-0.1cm} \min \Bigg\{ \rho_j \Big( s_j,\delta_j(s_j) \Big), \min_{\mathcal{T} \in \mathcal{T}_{i-1}} \hspace{-0.2cm} \max \set{  \min_{t=j+1}^{H} \rho_t \Big( s_t,\delta_t(s_t) \Big), 1- \min_{t=j}^{H-1} \pi_t \Big( s_{t+1} \Big\vert s_t , \delta_t(s_t) \Big) \hspace{-0.14cm} } \hspace{-0.1cm} \Bigg\} \\
\nonumber & = &  \displaystyle \max_{(\delta) \in \Delta_i} \min_{s_{j+1} \in \mathcal{S}} \min \Bigg\{ \rho_j \Big( s_j,\delta_j(s_j) \Big), \min_{\mathcal{T} \in \mathcal{T}_{i-1}} \hspace{-0.2cm} \max \set{  \min_{t=j+1}^{H} \rho_t \Big( s_t,\delta_t(s_t) \Big), \max_{t=j}^{H-1} \set{ 1 -  \pi_t \Big( s_{t+1} \Big\vert s_t , \delta_t(s_t) \Big) } } \Bigg\} \\
\nonumber & = &  \displaystyle \max_{(\delta) \in \Delta_i} \min_{s_{j+1} \in \mathcal{S}} \min \Bigg\{ \rho_j \Big(s_j,\delta_j(s_j)\Big), \max \bigg\{ 1- \pi_j \Big( s_{j+1} \Big\vert s_j, \delta_j(s_j) \Big), \\ 
\nonumber & & \hspace{4.5cm} \min_{\mathcal{T} \in \mathcal{T}_{i-1}} \max \Big\{  \min_{t=j+1}^{H} \rho_t \Big( s_t,\delta_t(s_t) \Big), 1- \min_{t=j+1}^{H-1} \pi_t \Big( s_{t+1} \Big\vert s_t , \delta_t(s_t) \Big) \Big\} \bigg\} \Bigg\} \\
\label{maxminminmax} & = &  \displaystyle \max_{\delta_j(s_j) \in \mathcal{A}} \min_{s_{j+1} \in \mathcal{S}} \min \Bigg\{ \rho_j \Big( s_j,\delta_j(s_j) \Big), \max \bigg\{ 1 - \pi_j \Big( s_{j+1} \Big\vert s_j, \delta_j(s_j) \Big), \\ 
\nonumber & & \hspace{3.5cm} \max_{(\delta) \in \Delta_{i-1}} \min_{\mathcal{T} \in \mathcal{T}_{i-1}} \max \Big\{  \min_{t=j+1}^{H} \rho_t \Big( s_t,\delta_t(s_t) \Big), 1- \min_{t=j+1}^{H-1} \pi_t \Big( s_{t+1} \Big\vert s_t , \delta_t(s_t) \Big) \Big\} \bigg\} \Bigg\} \\
\nonumber & = &  \displaystyle \max_{a \in \mathcal{A}} \min_{s_{j+1} \in \mathcal{S}} \min \Bigg\{ \rho_j \Big(s_j,a \Big), \max \bigg\{ 1 - \pi_j \Big( s_{j+1} \Big\vert s_j, a \Big),\underline{U^*_{H-1}}(s_{j+1}) \bigg\} \Bigg\} \\
\nonumber & = &  \displaystyle \max_{a \in \mathcal{A}} \min \Bigg\{ \rho_j \Big(s_j,a\Big), \min_{s_{j+1} \in \mathcal{S}} \max \bigg\{ 1 - \pi_j \Big( s_{j+1} \Big\vert s_j, a \Big),\underline{U^*_{H-1}}(s_{j+1}) \bigg\} \Bigg\} .
\end{eqnarray}
where the equation (\ref{equationmaxmin7}) of Property \ref{property_minmax} is used for the row (\ref{maxminlambdafg}).
The row (\ref{maxminminmax}) is explained by the following result:
consider the function $F: \mathcal{S} \times \mathcal{A} \rightarrow \mathcal{L}$.
Then,
\[ \max_{\delta: \mathcal{S} \rightarrow \mathcal{L}} \min_{s\in \mathcal{S}} F(s,\delta(s)) = \min_{s \in \mathcal{S}} \max_{\delta: \mathcal{S} \rightarrow \mathcal{A}} F(s, \delta(s)). \]
Indeed, $\forall \delta: \mathcal{S} \rightarrow \mathcal{A}$, $\forall s \in \mathcal{S}$,
$\displaystyle \min_{s' \in \mathcal{S}} F(s',\delta(s')) \leqslant \max_{\delta': \mathcal{S} \rightarrow \mathcal{A}} F(s,\delta'(s))$,
and thus, $\displaystyle \max_{\delta: \mathcal{S} \rightarrow \mathcal{A}} \min_{s' \in \mathcal{S}} F(s',\delta(s')) \leqslant \min_{s \in \mathcal{S}} \max_{\delta': \mathcal{S} \rightarrow \mathcal{A}} F(s,\delta'(s))$.
Now, consider $\delta^*: \mathcal{S} \rightarrow \mathcal{A}$
such that $\forall s \in \mathcal{S}$, $\delta^*(s) \in \operatorname*{argmax}_{a \in \mathcal{A}} F(s,a)$.
Then, $\forall \delta: \mathcal{S} \rightarrow \mathcal{A}$,
$\displaystyle \min_{s \in \mathcal{S}} F(s,\delta^*(s)) \geqslant \min_{s \in \mathcal{S}} F(s, \delta(s))$,
which implies that $\displaystyle \min_{s \in \mathcal{S}} F(s,\delta^*(s)) \geqslant \max_{\delta: \mathcal{S} \rightarrow \mathcal{A}} \min_{s \in \mathcal{S}} F(s, \delta(s))$,
\textit{i.e.} $\displaystyle \min_{s \in \mathcal{S}} \max_{\delta: \mathcal{S} \rightarrow \mathcal{A}} F(s,\delta(s)) \geqslant \max_{\delta: \mathcal{S} \rightarrow \mathcal{A}} \min_{s \in \mathcal{S}} F(s, \delta(s))$
: the equality is shown.

Finally, the strategy computation (\ref{equation_recursivepessstrat})
is explained by inclusions (\ref{equationargmax1}) and (\ref{equationargmax2})
of properties (\ref{property_minmax}).
\end{proof}







\subsection{Proof of Theorem \ref{belief_process_recursif_poss}}
\label{belief_process_recursif_poss_RETURN}
\begin{proof}
Suppose that the belief state at time step $t \in \mathbb{N}$
is $\beta_t$.
The joint distribution of the system state variable $S_{t+1}$
and the observation variable $O_{t+1}$ is
\\
$\Pi \paren{ S_{t+1} = s', O_{t+1} = o' \sachant I_{t} = i_{t}, a }$
\begin{align}
\label{proof_belupdate_1} &= \max_{s \in \mathcal{S}} \Pi \paren{ S_{t+1} = s', O_{t+1}= o', S_{t} = s \sachant I_{t} = i_{t}, a_t }\\
\label{proof_belupdate_2} &= \max_{s \in \mathcal{S}} \min \Big\{ \Pi \paren{ S_{t+1} = s', O_{t+1}= o' \sachant S_{t} = s, I_{t} = i_{t}, a_t }, \Pi \paren{ S_{t} = s \sachant I_{t} = i_{t}, a_t } \Big\}\\
\label{proof_belupdate_3} & = \max_{s \in \mathcal{S}} \min \Big\{ \Pi \paren{O_{t+1}= o' \sachant S_{t+1} = s', a_t}, \Pi \paren{ S_{t+1} = s' \sachant S_{t} = s, a_t }, \beta_t(s) \Big\}\\
\label{proof_belupdate_4} & = \min \set{ \Pi \paren{O_{t+1}= o' \sachant S_{t+1} = s', a_t},  \max_{s \in \mathcal{S}} \min \Big\{ \Pi \paren{ S_{t+1} = s' \sachant S_{t} = s, a_t }, \beta_t(s) \Big\} }\\
\nonumber & = \min \set{ \pi_{t} \paren{o' \sachant s', a_t},  \max_{s \in \mathcal{S}} \min \Big\{ \pi_t \paren{ s' \sachant s, a_t }, \beta_t(s) \Big\} },
\end{align}
denoted by $\pi_t \paren{ s', o' \sachant \beta_t, a_t}$ to simplify notations:
$\max_{s' \in \mathcal{S}} \pi_t \paren{ s', o' \sachant \beta_t, a_t}$ is also denoted by $\pi \paren{ o' \sachant \beta_t, a_t}$.
Line (\ref{proof_belupdate_1}) is the possibilistic marginalization over variable $S_t$. 
Line (\ref{proof_belupdate_2}) is due to the definition of the conditioning, Definition \ref{def_cond}. 
Line (\ref{proof_belupdate_3}) uses the Definition of the belief state, Definition \ref{def_QualPossBel},
and that $S_t$ does not depend on the action $a_t$.
Finally, line (\ref{proof_belupdate_4}) is comes from equation (\ref{equationmaxmin3}) of Property \ref{property_minmax}.

Suppose now that the observation received at time step $t+1$ is $o_{t+1}$:  
as, by definition, $\beta_{t+1}(s') = \Pi \paren{ S_{t+1} = s' \sachant I_{t} = i_{t}, O_{t+1} = o_{t+1}, a_t }$,
using the qualitative possibilistic conditioning (Definition \ref{def_cond}),
we conclude that, $\forall s \in \mathcal{S}$,
\begin{align*}
\beta_{t+1}(s') &= \left \{ \begin{array}{ccc}
1 & \mbox{ if } \pi_t \paren{ s', o_{t+1} \sachant \beta_t, a_t}= \pi_t \paren{ o_{t+1} \sachant \beta_t,a_t }, \\
\pi_t \paren{ s', o_{t+1} \sachant \beta_t, a_t} & \mbox{ otherwise. } 
\end{array} \right. 
\end{align*}
\end{proof}










\subsection{Proof of Theorem \ref{piPOMDPrewriting}}
\label{piPOMDPrewriting_RETURN}
\begin{proof}
Let us denote by $\pi \Big( s_H, \widehat{o}_H  \Big\vert \beta_0, (\delta) \Big)$
the joint possibility degree of the last system state $s_H \in \mathcal{S}$
and the observation sequence $\widehat{o}_H$ when the strategy is $(\delta) = (\delta)_{t=0}^{H-1}$:\\
$\Pi \paren{ S_H = s_H, \widehat{O}_H = \widehat{o}_H  \sachant (\delta) }$
\[  =\max_{(s_0,\ldots,s_{H-1}) \in \mathcal{S}^H} \min_{t=0}^{H-1} \min \set{ \pi_t \Big( o_{t+1} \Big\vert s_{t+1}, \delta_t(i_t) \Big), \pi_t \Big( s_{t+1} \Big\vert s_t,\delta_t(i_t) \Big), \beta_0(s_0)}. \]
The possibility distribution over the observation sequence is also denoted by
\[ \pi \Big( \widehat{o}_H  \Big\vert \beta_0, (\delta) \Big) = \max_{s_H \in \mathcal{S}} \pi \Big( s_H, \widehat{o}_H  \Big\vert \beta_0, (\delta) \Big), \] 
and the one over the last state $s_H \in \mathcal{S}$ is denoted by
\[ \pi \Big( s_H \Big\vert \beta_0, (\delta) \Big) 
= \max_{\widehat{o}_H} \pi \Big( s_H, \widehat{o}_H  \Big\vert \beta_0, (\delta) \Big). \]
By definition, the qualitative belief state at the end of the execution, \textit{i.e.} at time step $t=H$,
is equal to $\beta^{\delta,\widehat{o}_H}_{\beta_0}(s) = \Pi \paren{ S_H = s \sachant \widehat{O}_H = \widehat{o}_H , \Big(\delta_t(i_t)\Big)_{t=0}^{H-1} }$,
see Definition \ref{def_QualPossBel}. 
It can be written as a function of $\widehat{o}_H$, $(\delta)$ and $\beta_0$,
and this belief state is then denoted by $\beta^{\delta,\widehat{o}_H}_{\beta_0}$:
\begin{align*}
\beta^{\delta,\widehat{o}_H}_{\beta_0}(s) &= \left\{ \begin{array}{ccc} 1 & \mbox{ if } \pi \Big( s, \widehat{o}_H \Big\vert \beta_0, (\delta) \Big) =   
\pi \Big( \widehat{o}_H \Big\vert \beta_0, (\delta) \Big), \\
\pi \Big( s, \widehat{o}_H \Big\vert \beta_0, (\delta) \Big) & \mbox{ otherwise, }
\end{array} \right.
\end{align*}
using the possibilistic conditioning, Definition \ref{def_cond}.

Following lines show the desired equality for the optimistic criterion: 
its rewriting, denoted by
$\displaystyle \mathbb{S}_{\Pi} \Big[ \max_{s \in \mathcal{S}} \min \set{\Psi(s), B^{\pi}_H(s)} \Big\vert \beta_0, (\delta) \Big]$,
is equal to
\begin{align}
\nonumber \max_{\widehat{o}_H} & \min \Bigg\{ \max_{s \in \mathcal{S}} \min \set{\Psi(s), \beta^{\delta,\widehat{o}_H}_{b_0}(s)}, \pi \Big( \widehat{o}_H \Big\vert \beta_0, (\delta) \Big) \Bigg\}\\
\label{proof_rewritingPOMDP1} &= \max_{\widehat{o}_H,s\in\mathcal{S}} \min \Big\{ \Psi(s), \beta^{\delta,\widehat{o}_H}_{b_0}(s), \pi \Big( \widehat{o}_H \Big\vert \beta_0, (\delta)\Big) \Big\}\\
\label{proof_rewritingPOMDP2} &= \max_{s\in\mathcal{S}} \min \Bigg\{ \Psi(s), \max_{\widehat{o}_H} \min \set{ \Pi \paren{ S_H = s \sachant \widehat{O}_H = \widehat{o}_H , \Big(\delta_t(i_t)\Big)_{t=0}^{H-1} }, \pi \Big( \widehat{o}_H \Big\vert \beta_0, (\delta) \Big) } \Bigg\}\\
\label{proof_rewritingPOMDP3} &= \max_{s\in\mathcal{S}} \min \bigg\{ \Psi(s), \pi \Big( s \Big\vert \beta_0, (\delta) \Big)  \bigg\}\\
\label{proof_rewritingPOMDP4} &= \mathbb{S}_{\Pi} \croch{ \Psi(S_H) \sachant \beta_0,(\delta) }.
\end{align}
Line (\ref{proof_rewritingPOMDP1}) comes using equality (\ref{equationmaxmin3}) 
of Property \ref{property_minmax}. 
Line (\ref{proof_rewritingPOMDP2}) uses the definition of the possibilistic belief state.
Line (\ref{proof_rewritingPOMDP3}) uses the definition of $\pi \Big( \widehat{o}_H \Big\vert \beta_0, (\delta) \Big)$,
and the possibilistic conditioning.
Finally, line (\ref{proof_rewritingPOMDP4}) is a notation: 
it is used
as the Sugeno integral 
is based on the distribution $\pi \Big( s_H \Big\vert \beta_0, (\delta) \Big)$.

As regards the pessimistic criterion (\ref{piPOMDPpessrewriting}),
the rewriting is shown in the same way:\\
$\mathbb{S}_{\mathcal{N}} \Big[ \min_{s \in \mathcal{S}} \max \set{\Psi(s), 1-B^{\pi}_H(s)} \Big\vert \beta_0, (\delta) \Big]$
\begin{align} 
\nonumber &= \min_{\widehat{o}_H} \max \Bigg\{ \min_{s \in \mathcal{S}} \max \set{\Psi(s), 1- \beta^{\delta,\widehat{o}_H}_{b_0}(s)}, 1- \pi \Big( \widehat{o}_H \Big\vert \beta_0, (\delta) \Big) \Bigg\}\\
\label{proof_rewritingPOMDPpess1} &= \min_{\widehat{o}_H,s\in\mathcal{S}} \max \bigg\{ \Psi(s), 1 - \beta^{\delta,\widehat{o}_H}_{b_0}(s), 1 - \pi \Big( \widehat{o}_H \Big\vert \beta_0, (\delta) \Big) \bigg\}\\
\label{proof_rewritingPOMDPpess2} &= \min_{s\in\mathcal{S}} \max \bigg\{ \Psi(s), \min_{\widehat{o}_H} \max \set{ 1 - \Pi \paren{ S_H =  s \sachant \widehat{O}_{H} = \widehat{o}_{H}, \Big(\delta_t(i_t)\Big)_{t=0}^{H-1} }, 1 - \pi \Big( \widehat{o}_H \Big\vert \beta_0, (\delta) \Big) } \bigg\}\\
\label{proof_rewritingPOMDPpess3} &= \min_{s\in\mathcal{S}} \max \bigg\{ \Psi(s), 1 - \max_{\widehat{o}_H} \min \set{ \Pi \paren{ S_H =  s \sachant \widehat{O}_{H} = \widehat{o}_{H}, \Big(\delta_t(i_t)\Big)_{t=0}^{H-1} }, \pi \Big( \widehat{o}_H \Big\vert \beta_0, (\delta) \Big) } \bigg\}\\
\label{proof_rewritingPOMDPpess4} &= \min_{s\in\mathcal{S}} \max \bigg\{ P(s), 1 - \pi \Big( s \Big\vert \beta_0, (\delta) \Big)  \bigg\}\\
\label{proof_rewritingPOMDPpess5} &= \mathbb{S}_{\mathcal{N}} \croch{ P(S_H) \sachant \beta_0, (\delta) }.
\end{align}
Line (\ref{proof_rewritingPOMDPpess1}) comes using the equality (\ref{equationmaxmin4}) 
of Property \ref{property_minmax} and
line (\ref{proof_rewritingPOMDPpess2}) uses the definition of the possibilistic belief state.
Line (\ref{proof_rewritingPOMDPpess3}) uses the equatity (\ref{equationmaxmin1}) and the equality (\ref{equationmaxmin2}) of Property \ref{property_minmax}.
Line (\ref{proof_rewritingPOMDPpess4}) uses the definition of $\pi \Big( \widehat{o}_H \Big\vert \beta_0, (\delta) \Big)$,
and the possibilistic conditioning. 
Finally, as the Sugeno integral 
is based on the distribution $\pi \Big( s_H \Big\vert \beta_0, (\delta) \Big)$,
the result can be denoted as 
line (\ref{proof_rewritingPOMDPpess5}).
\end{proof}

\subsection{Proof of Theorem \ref{theorem_qualpossMarkov}}
\label{theorem_qualpossMarkov_RETURN}
\begin{proof}
First, $\forall a_t \in \mathcal{A}, \forall o' \in \mathcal{O}$,
\begin{align*}
\Pi \paren{ O_{t+1}=o' \sachant I_t = i_t, a_t } &= \max_{s' \in \mathcal{S}} \Pi \paren{ O_{t+1} = o', S_{t+1} = s' \sachant I_t = i_t, a_t } \\
&= \max_{(s,s') \in \mathcal{S}^2} \min \Big\{ \pi_t \paren{ o' \sachant s', a_t }, \pi_t \paren{ s' \sachant s, a_t }, \Pi \paren{ S_t=s \sachant I_t=i_t } \Big\} \\
&= \max_{(s,s') \in \mathcal{S}^2} \min \Big\{ \pi_t \paren{ o' \sachant s', a_t }, \pi_t \paren{ s' \sachant s, a_t }, \beta_{\beta_0}^{i_t}(s) \Big\}.
\end{align*}
where $\beta_{\beta_0}^{i_t}$ 
is obtained starting from $\beta_0$ 
and with information $i_t$.
To make the next equations clear,
the next formula
$\max_{(s,s') \in \mathcal{S}^2} \min \set{ \pi_t \paren{ o' \sachant s', a_t }, \pi_t \paren{ s' \sachant s, a_t }, \beta_{\beta_0}^{i_t}(s) }$
is denoted by $\pi_t \paren{ o' \sachant \beta_{\beta_0}^{i_t}, a_t }$.
Then, as the belief state is a deterministic function of the current observation, the previous action, 
and the previous belief state,
\begin{align}
\nonumber \Pi \paren{ B^{\pi}_{t+1} = \beta' \sachant I_t=i_t, a_t } &= \Pi \paren{ \cup_{\substack{ o' \in \mathcal{O} \mbox{ \tiny s.t. } \\ \nu(\beta,a_t,o') = \beta'}} \set{ O_{t+1} = o'} \sachant I_t=i_t, a_t } \\
\nonumber&= \max_{ \substack{ o' \in \mathcal{O} \mbox{ \tiny s.t. } \\ \nu(\beta,a_t,o') = \beta'} } \Pi \paren{ O_{t+1}=o' \sachant I_t = i_t, a_t }  \\
\label{possibilistic_belief_trans} &= \max_{ \substack{ o' \in \mathcal{O} \mbox{ \tiny s.t. } \\ \nu(\beta,a_t,o') = \beta'} } \pi_t \paren{ o' \sachant \beta, a_t }.
\end{align}
where $\beta_{\beta_0}^{i_t}(s)$ is denoted by $\beta$.
The set of all the possible information sequences at a time step $t \geqslant 1$ 
is denoted by $\mathcal{I}_{t} = \mathcal{A}^{t} \times \mathcal{O}^{t}$.
Thanks to the equation (\ref{possibilistic_belief_trans}) 
for each belief $\beta \in \Pi^{\mathcal{S}}_{\mathcal{L}}$,
$\Pi \paren{ B^{\pi}_{t+1} = \beta' \sachant I_t=i_t, a_t }$ 
does not depend on the information $i_t \in \mathcal{I}_t$, 
provided $i_t$ is in $\set{ i \in \mathcal{I}_t \sachant \beta^{i}_{\beta_0} = \beta }$.
Thus, as $\Pi \paren{ I_t = i \sachant B^{\pi}_t=\beta, a_t } = 0$ if $\beta_{\beta_0}^{i} \neq \beta$,
we can write \\
\\
$ \displaystyle \Pi \paren{ B^{\pi}_{t+1} = \beta' \sachant B^{\pi}_{t}=\beta, a_t }$
\begin{align*}
 &= \max_{i \in \mathcal{I}_{t}} \min \Big\{ \Pi \paren{ B^{\pi}_{t+1} = \beta' \sachant I_t=i, a_t }, \Pi \paren{ I_t = i \sachant B^{\pi}_t=\beta, a_t } \Big\}\\
 &= \max_{i \in \mathcal{I}_{t}} \min \Big\{ \Pi \paren{ B^{\pi}_{t+1} = \beta' \sachant I_t=i^*, a_t }, \Pi \paren{ I_t = i \sachant B^{\pi}_t=\beta, a_t } \Big\}\\
&= \min \set{ \Pi \paren{ B^{\pi}_{t+1} = \beta' \sachant I_t=i^*, a_t }, \max_{i \in \mathcal{I}_{t}} \Pi \paren{ I_t = i \sachant B^{\pi}_t=\beta, a_t } }\\
&=  \Pi \paren{ B^{\pi}_{t+1} = \beta' \sachant I_t=i^*, a_t }
\end{align*}
where $i^*$ is such that $\beta_{\beta_0}^{i^*} = \beta$, 
using the equation (\ref{equationmaxmin3}) of Property \ref{property_minmax},
and the fact that $\max_{i \in \mathcal{I}_{t}} \Pi \paren{ I_t = i \sachant B^{\pi}_t=\beta, a_t } = 1$.
\end{proof}


