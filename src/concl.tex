%%%fisherman's wharf hostel + bike
\chapter*{Conclusion}
\addcontentsline{toc}{chapter}{Conclusion}
Contributions of this thesis are
mainly related to the preliminary work 
of R\'egis Sabbadin \cite{Sabbadin:1999:pipomdp}.
The latter proposes 
a possibilistic counterpart 
of the POMDPs 
modeling uncertainty 
with qualitative possibility distributions.
%%% CONCL INTRO
Hence, in our work, 
qualitative possibilistic models 
for planning under uncertainty
are developed and studied. 
This study is meant to adress some issues 
of the probabilistic POMDPs 
detailed in Introduction: 
the probabilistic framework is also 
formally introduced 
in the first chapter.

The major motivation of this study 
is the complexity reduction: 
while the probabilistic belief space is infinite,
the possibilistic one is finite.
The qualitative possibilistic framework 
thus offers an appropriate 
belief space discretization.
Moreover, as the belief state 
is a possibility distribution 
over the system space in this framework, 
total ignorance can be defined 
by a belief equal to $1$ on all states.
If a given system state is perfectly known
to be the actual one, 
the belief state equal to $1$ on this state 
and to $0$ on all other states 
is an appropriate representation.

Imprecisions of the probability distributions
are also naturally encoded 
with the possibilistic formalism, 
resulting in two criteria
for the selection of the strategy:
one is optimistic 
and the other pessimistic.
A more practical advantage is that 
the qualitative possibilistic modeling 
needs less information about the system
than the probabilistic one:
the plausibilities of events 
are ``only'' classified 
in the possibilistic scale $\mathcal{L}$ 
but not quantified.
Possibilistic models can be seen 
as a tradeoff between \textit{non-deterministic} ones, 
whose uncertainties are not at all quantified 
yielding a more imprecise model, 
and probabilistic ones, 
where uncertainties are fully specified.
Indeed, under the non-deterministic formalism,
an elementary event is either ``possible'',
or ``impossible'':
no degree is available
to differentiate a highly plausible event 
from an unlikely one.
The resolution of planning problems 
using the framework of non-determinism 
is called \textit{contingent/conformant planning} 
studied for instance in \cite{Albore_atranslation-based,bonet2014flexible}.
By the way, in the introduction of this thesis,
we might also add ``non-determinism''
as a particular case of Possibility Theory
in Figure \ref{uncertainty_theories},
as values of associated non additive measures
are $0$ or $1$ instead of a more flexible scale $\mathcal{L}$.

In a nutshell, 
this thesis consists in
theoretical and practical contributions:
on the one hand, theoretical contributions 
are for instance 
the proposed updates of the qualitative possibilistic processes
-- mixed-observability and management of unbounded executions -- 
or independence results on them, with associated proofs.
On the other hand, 
practical contributions
are the demonstration of the accuracy 
of qualitative possibilistic models
to simplify computations or for modeling,
via experimental results (e.g. IPPC 2014) 
and modeling examples (e.g. chapter on human-machine interaction).
Particular emphasis is being placed 
on the motivations developed in Introduction
besides complexity reduction and problem imprecisions:
namely, robotic applications and belief management.
For instance, target recognition missions 
are studied in the second and third chapters 
(e.g. \textit{Rocksample} problem, 
or even the mission described in Figure \ref{robotgridfig});
% robot chap2; robot reconnaissance chap navig/rocksampple PPUDD-IPPC-robotic system 
IPPC 2014 contains also problems comparable to robotic systems 
(e.g. \textit{Elevator} problem, or \textit{Tamarisk} problem, also used in 2014 RL competition). 
Moreover, the fourth chapter shows good results 
in estimating human assessment with qualitative possibility distributions
(\textit{i.e.} belief states on the human assessment of the machine state).
Finally, the last contribution, 
\textit{i.e.} the hybrid probabilistic-possibilistic POMDP, 
can be seen as a concluding work, taking into account
potential issues when using purely possibilistic models 
for planning under uncertainty.
A more detailed review of the contributions follows.

\subsection*{New features for the $\pi$-POMDPs
and first strategy executions}
%%%%%%%%%%%%%
%%% chap2 %%%
%%%%%%%%%%%%%
%%THEORETICAL CONTRIBUTIONS
The very first update 
of the Qualitative possibilistic POMDPs \cite{Sabbadin:1999:pipomdp} 
concerns the qualitative aggregations 
of the preferences over time:
we first show that, 
much like in the probabilistic framework,
preference aggregation can be derived 
from the properties of the framework 
-- here the qualitative counterpart 
of the linearity
for Sugeno integrals for instance --
as proved in Annex \ref{theorem_intermpref_RETURN} 
using the Theorem \ref{piPOMDPoptrewriting} 
about belief-dependent value function.
The latter is also a contribution 
as the first formal construction
of the $\pi$-POMDP model.
Different approaches between the pessimistic and the optimistic criterion
are also presented: 
note that the mixed optimistic-pessimistic criterion 
(see Definition \ref{def_optpesscrit}) 
is mainly used along our work 
since it is equivalent to 
a $\pi$-MDP with an optimistic criterion.
As the optimistic criterion is compatible 
with proposed algorithms for time-unbounded processes,
and produces often better strategies 
for the treated problems,
this allows to profit of these points.
The mixed optimistic-pessimistic criterion 
is pessimistic according to the belief state:
we observe that it produces a good tradeoff
with the optimistic global criterion.

We then adapted $\pi$-POMDPs 
to mixed-observability \cite{OngShaoHsuWee-IJRR10},
which takes part in the theoretical contributions of our work:
this contribution, called $\pi$-MOMDP, 
dramatically reduces the size of the belief space 
and thus allows the first computations of strategies from $\pi$-POMDPs. 
Finally, an other theoretical contribution 
is the qualitative counterpart 
of the value iteration algorithm,
with the associated criterion 
for time-unbounded executions. 
As proved in Annex \ref{theorem_DPpiMOMDP_RETURN}
(and our publication \cite{Drougard13}), 
there exists an optimal strategy, 
which is stationary 
\textit{i.e.} which does not depend on the stage of the process $t$. 
This strategy can be computed 
by the proposed dynamic programming scheme: 
it is also shown that 
the number of iterations 
to make the value function converge 
is lower than the size of the state space.
The assumption of the existence of a ``stay'' action
is not a constraint in practice
as it is only selected 
in some goal states.
Note also that the target recognition missions 
presented in this thesis 
have typically unbounded durations.

%%%PRACTICAL ONES
As already pointed out,
the experiments in the second chapter 
use both previous theoretical contributions.
Indeed, the mixed-observability property of the problem,
makes computations feasible for our robotic example.
The proposed criterion is also really useful: 
it is convenient to allow 
the computation of strategies 
for robotic missions 
with unbounded durations.
For instance, in our example, 
it allows to define properly the mission: 
if the robot has not figured out 
which target is right one,
we want the mission to be continued.
The first practical contributions of this thesis 
is thus the computation of a strategy
for this robotic mission and its execution.
Indeed, we have shown 
that the qualitative possibilistic approach 
can outperform probabilistic POMDP ones 
for a target recognition problem 
where the agent's observations dynamics 
is not properly defined.
Note that the only information used 
to define system observation of the system,
is that the more the robot is far from a target, 
the more the observation is noisy:
it already shows that the possibilistic framework
may be useful in case of restricted knowledge 
about the problem.


%%%OBSERVATIONS
Finally, the second chapter highlights 
a very interesting behavior 
of the qualitative possibilistic belief state:
under some conditions, 
the belief update, 
which is defined from the counterpart of Bayes rule \cite{Dubois199023},
increases the knowledge associated to the belief state.
Indeed, in our example, 
the belief state is responsible to the imprecision 
as it only takes into account 
more reliable observations:
the next belief is either more skeptic 
about a state if its confirms the prior belief; 
or changes to the opposite belief if it contradicts it.
On the contrary a probabilistic belief 
is eventually modified at each step. 
Some conditions leading to this behavior 
are then presented,
and associated proofs are given 
(see Annex \ref{theorem_specificity_belief_RETURN}).

\subsection*{Graphical work on independence and factorization (PPUDD)}
%%%%%%%%%%%%%
%%% chap3 %%%
%%%%%%%%%%%%%
%%THEORETICAL CONTRIBUTIONS
%%CHAP3
Next theoretical contribution is 
the introduction of the factored $\pi$-MOMDPs:
indeed, we have considered processes
whose state space is represented by $n$ 
post-action independent variables,
\textit{i.e.} processes whose 
state variables of the same time step
are independent (conditional on the past).
The algorithm proposed to exactly solve 
these factorized processes,
called PPUDD, is another contribution:
it is the symbolic version of 
the dynamic programming scheme 
proposed in the previous chapter.
Inspired by SPUDD \cite{Hoey99spudd:stochastic}, 
PPUDD means \emph{Possibilistic Planning Using Decision Diagrams}. 
As SPUDD, it operates on ADDs 
encoding value, preference and transition functions.
Finally, the last theoretical contributions of the third chapter
are the proofs of independence results,
Theorem \ref{thmSHind} and Theorem \ref{thmVARind}:
they relate to a proposed graphical model
representing a particular factorization of the process
\textit{i.e.} particular independence assumptions
on the variables.
Thus, we have shown that 
these independence assumptions
lead to a natural factorization 
of the space of the belief states.
The belief state variables of a $\pi$-MOMDP 
satisfying these assumptions
are post-action independent,
and the associated problem 
can be more easily solved by PPUDD. 
The independence of sensors 
and of corresponding hidden state variables 
suffices to fulfill these conditions:
for instance, the \textit{RockSample} problem, 
well illustrates these conditions.

%%%% PRACTICAL CONTRIB
%% WEE!
The motivation leading to the conception of PPUDD,
\textit{i.e.} the guess that the use of operators $\min$ and $\max$ 
leads to smaller ADDs, and thus to faster computations,
has been verified in practice:
our experiments and the results 
of the International Probabilistic Planning Competition (IPPC 2014\footnote{\url{https://cs.uwaterloo.ca/~mgrzes/IPPC_2014/}})
show that this possibilistic approach
can involve lower computation time
and produce better policies
than its probabilistic counterparts
when computation time is limited,
or for high dimensional problems.
For instance, 
PPUDD performances have been compared 
to the probabilistic MOMDP solver APPL \cite{Kurniawati-RSS08,OngShaoHsuWee-IJRR10}
and the symbolic HSVI solver \cite{Sim:2008:SHS:1620163.1620241},
in terms of computation time, 
and using the average of 
the total reward at execution. 
PPUDD computes a strategy maximizing 
exactly the possibilistic criterion 
while APPL and symb. HSVI compute strategies
which approximately maximize the expected sum of rewards.
The experimental results on the Rocksample problem 
show that using an exact algorithm 
(PPUDD) for an approximate model ($\pi$-MOMDPs) 
can run significantly faster 
than reasoning about exact models, 
while providing better policies 
than approximate algorithms (APPL) 
for exact models.
Most of the cited contributions of this chapter
have been published in \cite{DBLP:conf/aaai/DrougardTFD14},
and an implementation of PPUDD 
to reproduce experiments 
can be found 
at the repository \url{www.github.com/drougui/ppudd}.

%% IPPC14
Finally, in order to focus on the behavior 
of the possibilistic qualitative approach 
in a wide panel of probabilistic problems, 
the next practical contribution 
is the participation 
in the fully observable track of IPPC 2014
with adapted versions of PPUDD.
The implementation of our solver for this competition 
was performed with the \textit{CU Decision Diagram Package}
for the ADDs computations. 
Comparing only solvers using ADDs, 
PPUDD and a probabilistic solver 
called symbolic LRTDP \cite{symbLRTDP}
(as a variant of \cite{Bonet03labeledrtdp}),
our solver produces better strategies 
than the probabilistic approach.

\subsection*{Discussion about the current results}
%% fin 3 lien CHAP4-5
%% ??  after? critique
Experiments concerning the Navigation problem
is a good illustration of the optimistic and pessimistic criteria
in the qualitative possibilistic framework: 
a robot has to reach a goal as soon as possible,
avoiding unsafe locations 
(for instance, locations where there is a risk that it falls down and break).
The strategy maximizing 
the optimistic criterion 
makes the robot 
unfrequently reach the goal,
but more quickly
than the one from the pessimistic criterion
which makes it however often reach the goal.  
Proposed approaches involve then 
the choice of a criterion.
This feature of the possibilistic approach
explains also the difficulties experienced
with the ``Traffic'' domain
of the competition: 
the optimistic criterion can be too optimistic
for the given problem, 
although strategies from it are generally 
more efficient than those from the pessimistic criterion
(that is why the optimistic criterion has been used for the competition).

Presented models 
assume also that preference
and uncertainty degrees
share the same scale
which makes it hard to
set in practice.
The fourth chapter about human-machine interaction
is meant to show that,
although qualitative possibilistic approach
may pose questions concerning modeling,
this approach can be very appropriate
in some practical situations,
as in our case, to produce diagnosis.

We got really good results 
with PPUDD in previous experiments 
among symbolic algorithms (SPUDD, symbolic LRTDP):
for instance, SPUDD provides good strategies, but 
cannot solve big instances of the tested problem. 
However, state space search algorithms
(PROST \cite{DBLP:conf/aips/KellerE12} and GOURMAND \cite{DBLP:conf/aaai/KolobovMW12}) won IPPC 2014,
and are yet far more efficient than ADD-based methods:
MDP instances with a complex structure
are encoded with ADDs big enough
to disqualify these approaches.
These observations and modeling considerations
are the motivations of the last chapter,
presenting the hybrid POMDP.


%% chap HMI 4
\subsection*{Estimating in a qualitative world (HMI)}
%% TODO
Obviously, although 
qualitative possibilistic models 
can lead to appropriate approximate strategies 
for high dimensional problems, 
they cannot lead to the best results
for simple problems
in purely frequentist worlds.
On the contrary, 
the fourth chapter focus on
the use of this framework
in a purely qualitative world:
indeed it proposes the study of
human-machine interactions
which only involve the deterministic 
behavior of the machine
and a qualitative expert knowledge
about the human behavior. 

This joint work with Sergio Pizziol 
shows that the use of the qualitative possibilistic processes
is an appropriate choice
when probabilities 
cannot be defined in practice.
Indeed, this framework produces
useful error detections and diagnosis
in real experiments 
with pilots in a simulator.

Finally, this work provides also 
theoretical contributions.
The first one is
summed up in Theorem \ref{thmUpdate} 
and Theorem \ref{estimHMI}.
These theorems lead to the definition
of the possibilistic estimation 
of the human assessment
within the framework of the qualitative possibilistic 
Hidden Markov Processes ($\pi$-HMPs):
this estimation is used to detect human attentional errors.
An other contribution 
is the diagnosis computation,
using the \textit{leximin} operator.
Finally, a toy example is also provided
to explain the use, the meaning and the behavior 
of these possibilistic tools in practice.

%%%%%%%%%%%%%%%%%%%%%%%%%
%%%%%%%%%%%%%%%%%%%%%%%%%
%%%%%%%%%%%%%%%%%%%%%%%%%
%% CHAP5 %% TODO
\subsection*{A possibilistic discretization of the belief space (hybrid POMDP}
The work on the hybrid POMDP is motivated 
by the proper discretization 
provided by the use 
of a qualitative possibilistic belief state.
The possibilistic nature of these belief states 
allows to define the reward attached to them 
in a pessimistic way using the Choquet integral.
This model is proposed as a way to compute
more easily strategies since the hybrid POMDP
is finally solved as a classical MDP.
This approach may also lead to cautious behaviors
as the reward definition on belief states is pessimistic
with respect to the lack of knowledge.

The computation of the transition function of the resulting MDP
are given: it uses a \textit{possibility-probability transformation}
called \textit{pignistic transformation} 
to recover a probability distribution 
from a possibilistic belief state, in a neutral way.

We described also how a factored POMDP 
leads to a factored and more tractable MDP:
indeed, final contribution shows how to profits from 
a large class of common structure (at least among IPPC domains)
to make the resulting MDP as simple as possible.
Hence, efficient MDP solvers such as PROST or GOURMAND, 
are able to solve the factored MDP 
representing the hybrid POMDP.
This model has been published in \cite{DBLP:conf/sum/DrougardDFT15}.
%As ADDs are still efficient for preprocessing,
%they can be easily used to translate any factored POMDP
%model into a factored MDP representing the hybrid POMDP.

%%%%%%%%%%%%%%%%%%%%%%%%
%%%%%%%%%%%%%%%%%%%%%%%%
%%%%%%%%%%%%%%%%%%%%%%%%
%%CONCL CONL
Note finally that all the partially observable processes presented in this work
(purely qualitative and hybrid)
can be translated into MDP or $\pi$-MDP
and are thus computable in finite time, 
at most exponential in the description of the problem.
\section*{Perspectives}
%%%
%%% PERSPECTIVES
%%%
First perspective comes from an observation:
the memory limits plausibly reached with the \textit{Triangle tireworld} 
problem of IPPC 2014 may be bypassed by a stronger discretization: 
versions of PPUDD of IPPC 2014 used a precision of $10^{-3}$
on the initial probabilistic model, leading to a big scale $\mathcal{L}$
-- defined in practice as all the transformed probability values and normalized reward	 values
present in the given MDP.
In any case, it would be instructive to have an idea
of the impact of the precision of the discretization
in the performances of PPUDD on probabilistic problems. 
More generally, more efforts 
on the translation 
from a probabilistic MDP into 
its possibilistic approximation
may significantly improve 
the approach we proposed for IPPC 2014.

As we presented the symbolic resolution of the $\pi$-MDPs (PPUDD), 
an heuristic-based strategy computation, 
like in \cite{DBLP:conf/aaai/Teichteil-KonigsbuchVI11},
may be a good contribution.
As regards reinforcement learning 
in the qualitative possibilistic context,
we can cite the following work \cite{DBLP:conf/fuzzIEEE/Sabbadin01}.

The second chapter does not propose any algorithm
to compute strategies 
for unbounded missions
according to the pessimistic criterion. 
The algorithm is straightforward, 
but the optimality of the resulting strategy 
for this criterion
seems really hard to prove.
However, an optimal strategy 
in this sense,
would be useful
in order to manage 
long-term unsafe problems. 

The independence results given in the third chapter
are valid for the probabilistic MOMDPs: 
it would be interesting to determine if it can have
a positive impact on the optimal strategy computation.

The work \cite{Bonet:2002:QMP:2073876.2073884} proposes an other framework
for planning under uncertainty, also qualified as qualitative. 
However, this work uses quantitative operations
as sum and product, and may be seen as a discretization of the probabilistic framework
(see \cite{Wilson:1995:OMC:2074158.2074221}).
The authors remark however that Qualitative Possibility Theory leads to 
criteria which have not enough information for discriminating
among optimal decisions.

Lots of works in Possibility Theory
may lead to more refined possibilistic MDPs
to improve modeling when needed.
We can mention the \textit{leximin} operator \cite{lexirefin},
used in the fourth chapter:
it avoid the drowning effect
of the operator $\min$ and $\max$.
It may be used for preference aggregation
or even to compute a refined possibility degree 
of a trajectory. 
It may be a useful improvement,
but note that it eventually makes 
the problem more complex.
Another update which has still to be performed
is The integration 
of more discriminative criteria 
\cite{LIP61723,conf/uai/GiangS01} 
to the $\pi$-MDP framework.

%CHAP4
In the work on HMI, 
the human operator is supposed to be certain 
about the state of the machine 
(even if her/his only guess may be wrong).
A future work could define 
the human assessment in a more complex way:
for instance as a possibility distribution
(a possibilistic belief state).

%CHAP5

Finally, the promising approach 
presented under the name hybrid POMDP
will be tested on the POMDPs 
of the IPPC competition \cite{SannerIPPC1111} in a future work:
indeed, the proposed problems
are factored POMDPs
as introduced in Section \ref{factorizationSection}.
