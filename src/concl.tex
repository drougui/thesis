\chapter*{Conclusion}
\addcontentsline{toc}{chapter}{Conclusion}


chap1: qualitative aggregations of the preferences over time
can be derived from properties of the qualitative framework,
and different approaches from pessimistic and optimistic.
contribution: preuve de convergence est alors n\'ecessaire dans le but de calculer une strat\'egie \cite{Drougard13}. 
puis une preuve est donn�e � l'homologue de l'algorithme d'it�ration sur les valeurs. 
can outperform 
probabilistic POMDP ones 
for a target recognition problem 
where the agent's observations dynamics 
is not properly defined.
sa mise \`a jour
poss\`ede la caract\'eristique int\'eressante d'accro\^itre la connaissance associ\'ee.


The experiment uses both previous contributions.
Indeed, the mixed-observability property of the problem,
makes computations feasible in our example.
The proposed criterion is also really useful: 
it is convenient to allow the computation of strategies 
for robotic missions with unbounded durations.
For instance, in our example, 
it allows to define properly the mission: 
if the robot has not figured out the right target,
we want the mission to be continued.




expe PO: robot chap2; robot chap3 PPUDD; chap4 HMI (belief on the human assessment);

Our experiments and the  results 
of the International Probabilistic Planning Competition (IPPC 2014)
show that this possibilistic approach
can involve lower computation time
and produce better policies
than its probabilistic counterparts.

Ces tests sur des probl\`emes de planification vari\'es m\`enent \`a l'observation que 
ces m\'ethodes \`a ADDs, dites \textit{symboliques}, ne font pas le poids face aux
approches proc\'edant \`a une recherche heuristique dans l'espace d'\'etat \cite{DBLP:conf/aips/KellerE12} (en termes de rewards... expliquer!). 
Nous avons pu aussi prendre note des inconv\'enients du formalisme possibiliste qualitatif pour la mod\'elisation:
le crit\`ere utilis\'e \'etant global, son choix restreint la g\'en\'eralit\'e du mod\`ele.\\




perspec: \cite{LIP61723} more discriminative criteria
PSR \ref{Littman01predictiverepresentations}

Through this thesis, we developped \dots
updates of possibilistic MDPs, 
modelization examples, 
demonstration of the accuracy of qualitative models,
graphical results (factorization),
simplification,
perspective in a hybrid model

\section*{Perspectives}
test hybrid POMDP\\
refined ranking\\
prob $\rightarrow$ poss\\

%Dans la  continuit� directe  de notre travail  de th�se,  nous pouvons
%\dots

online POMDP solving (ref caro)

The work \cite{Bonet:2002:QMP:2073876.2073884} proposes an other framework
for planning under uncertainty, also qualified as qualitative. 
However, this work uses quantitative operations
as sum and product, and may be seen as a discretization of the probabilistic framework
(see \cite{Wilson:1995:OMC:2074158.2074221}).
The authors remark however that Qualitative Possibility Theory leads to 
criteria which have not enough information for discriminating
among optimal decisions.

\begin{quote}
Providing the autonomy to a robot 
consists in computing a function 
which returns accurate robot actions
over time given the system observations. 
Partially Observable Markov Decision Processes (POMDPs) 
is a well-suited model 
for such purpose 
modeling probabilistic dynamics 
of the system.
The use of the POMDPs 
raises however some practical issues,
such as the difficulty to encode 
robot ignorance, 
or the high complexity 
of the strategy computation problem. 
This paper presents recent contributions 
in the use of the Qualitative Possibility Theory
for planning under uncertainty, 
studied to answer to these concerns.
It begins with an update 
of the work of Sabbadin 
about a possibilistic counterpart 
of POMDPs called $\pi$-POMDP. 
Then strategies computation time is decreased 
with the use of Agebraic Decision Diagrams (ADDs)
and benefiting from the problem structure.
We compare performances 
of our solvers with those of 
its probabilistic counterparts, 
in terms of computation time, 
and with criteria measuring 
the mission achievement. 
While this possibilistic framework 
provides good results,
some highlighted issues
are finally discussed: 
conclusion argues for a hydrid POMDP 
model with both probabilistic 
and possibilistic settings. 
\end{quote}
\section*{Introduction}
Partially Observable Markov Decision Processes 
(POMDPs) \cite{Smallwood_Sondik} 
define a useful formalism 
to express sequential decision problems 
under uncertainty.
This framework models 
an agent and its mission, 
without assuming 
that the latter knows directly 
the current state of the system.
In our study, 
the agent is more precisely a robot, 
and the system state 
consists of a description of the situation, 
including robot and environment features
of interest.
A POMDP is defined 
setting up a set of possible system states 
$s \in \mathcal{S}$,
and reward values $r \in \mathbb{R}$ on it, 
modeling the robotic mission. 
A full definition of this process 
includes as well the set of 
possible observations of the system, 
$o \in \mathcal{O}$, 
and the set of available actions for the robot, 
$a \in \mathcal{A}$. 
Each action redefines 
the conditional probabilities 
linking system states over time, 
namely $\textbf{p} \paren{s' \sachant s, a}$, 
and those defining uncertainty of observations 
conditionned on the actual system state, 
$\textbf{p} \paren{ o' \sachant s', a}$:
actions have a stochastic control on system states 
and on system observations, 
and the aim is to make action choices 
leading to states with greatest rewards.
Solving a POMDP consists in computing 
a function called \textit{strategy} 
which returns a proper action at each process step, 
according to all received observations and selected actions  
\textit{i.e.} all of the data available for the robot.
A probability distribution known as \textit{belief} 
estimates the actual system state, 
as the latter is not directly observable. 
The belief is updated at each stage of the process 
using the current observation.
It can be shown that an optimal strategy
may be chosen among functions of the belief.
Figure \ref{ID_pomdp} is a graphical representation 
of a POMDP called \textit{Influence Diagram}.
\begin{figure}
\label{ID_pomdp}
\caption{Influence Diagram of a POMDP, with system states $s \in \mathcal{S}$, 
observations $o \in \mathcal{O}$, actions $a \in \mathcal{A}$, 
rewards encoding mission goal $r \in \mathbb{R}$, 
and agent beliefs $b$.}
\end{figure}
 
Consider now robots whose observations 
come from computer vision algorithms based on statistical learning: 
perception performances depend on the training dataset of pictures,
and observation probability distributions $\textbf{p} \paren{ o' \sachant s',a }$
are also frequencies computed using such a labeled dataset.
The imprecision about 
the actual vision algorithm behaviour
with real world pictures
has to be taken into account 
to make the robot autonomous 
under any circumstances. 
Moreover, consider missions 
where a part of the system state,
describing something that the robot
is supposed to infer by itself, 
is initially fully unknown:
for instance the location 
or the nature of a target. 
Classical approaches initialize 
the belief as 
a subjective uniform probability distribution. 
In this case the belief update 
mixes subjective probabilities and frequencies,
which is questionable.
Finally solving a POMDP 
\textit{i.e.} computing an optimal strategy 
is at least a PSPACE-complete problem. 
In practice, optimality can be reached 
for tiny problems, 
or highly structured ones. 
Otherwise, only approximate solutions can be computed.  
The use of the Qualitative Possibility Theory \cite{DBLP:journals/eor/DuboisPS01}
is studied here,
as it appears capable
to both simplify the POMDPs, 
and model imprecision and ignorance 
related to robotic missions.

Qualitative possibilistic POMDPs, 
$\pi$-POMDPs \cite{Sabbadin:1999:pipomdp}, 
are alternative processes 
defined using a qualitative evaluation 
of events plausibility
instead of probabilities: 
it allows to formally represent 
agent ignorance, 
and imprecision on observations hazard. 
In this paper, 
a possibilistic version of 
Mixed-Observable MDPs \cite{OngShaoHsuWee-IJRR10}, 
called $\pi$-MOMDP \cite{Drougard13}, 
is first presented to reduce dramatically 
the complexity of solving $\pi$-POMDPs, 
some state variables of which are fully observable. 
An algorithm
for missions with unbounded durations
is next proposed:
returned strategies can outperform 
probabilistic POMDP ones 
for a target recognition problem 
where the agent's observations dynamics 
is not properly defined.

Then factorized $\pi$-MOMDPs
are defined making possible the processing
of large structured planning problems. 
Building upon the probabilistic 
SPUDD algorithm \cite{Hoey99spudd:stochastic},
we conceived an algorithm named PPUDD \cite{DBLP:conf/aaai/DrougardTFD14} 
for solving factorized $\pi$-MOMDPs
using \textit{Algebraic Decision Diagrams} (ADD).
Our experiments and the  results 
of the International Probabilistic Planning Competition (IPPC 2014)
show that this possibilistic approach
can involve lower computation time
and produce better policies
than its probabilistic counterparts:
it highlights also some issues 
of these qualitative models.
Finally, points raised are added to 
the symbolic computations limits,
and lead to the description
of a future work.
\section*{Background}
The work of Sabbadin \cite{Sabbadin:1999:pipomdp} 
proposes a possibilistic counterpart 
of the POMDPs which models the uncertainty with 
\textit{qualitative possibility distributions}.
The \textit{possibility scale}, 
is defined as 
$\mathcal{L} = \{0,\frac{1}{k}, \frac{2}{k}, \ldots, \frac{k-1}{k}, 1 \}$ 
with $k \in \mathbb{N}^*$.
A qualitative possibility distribution over $\mathcal{S}$ 
is a function $\pi: \mathcal{S} \rightarrow \mathcal{L}$ 
such that $\max_{s \in \mathcal{S}} \pi(s) = 1$ 
(possibilistic normalization).
Inequality $\pi(s)<\pi(s')$ means that state $s'$ 
is more plausible than state $s$.
This modeling needs less information 
than the probabilistic one since the plausibilities of events 
are ``only'' classified in $\mathcal{L}$ but not quantified.
 
The transition function $T^{\pi}$ describes
the uncertainty over the next state: for each $ (s,s') \in \mathcal{S}^2$, $a \in \mathcal{A}$, 
$T^{\pi}(s,a,s') = \pi \paren{ s'\sachant s,a } \in \mathcal{L}$ is
the possibility of reaching the system state $s'$ 
conditionned on the current state $s$ and action $a$.
Scale $\mathcal{L}$ serves as well to model 
the \emph{preference} over states: 
function $\mu: \mathcal{S} \rightarrow \mathcal{L}$ assigns a degree
at each system state modeling how well this state fulfill the mission. 
The uncertainty over observations 
is encoded in the observation function 
$\Omega^{\pi}$: 
$\forall o' \in \mathcal{O}$, $s \in \mathcal{S}$, $a \in \mathcal{A}$, 
$\Omega^{\pi} \paren{ s',a,o' } = \pi \paren{ o' \sachant s,a }$ 
is the possibility of the current observation $o'$ 
conditionned on the current state $s'$ and the previous action $a$. 

A $\pi$-POMDP is fully defined by the tuple 
$\langle \mathcal{S},\mathcal{A},\mathcal{L}, T^{\pi}, 
\mathcal{O}, \Omega^{\pi}, \mu,\beta_0 \rangle$,
where $\beta_0$ is the initial possibilistic belief. 
The agent belief is a possibility distribution 
over $\mathcal{S}$: total ignorance is defined 
by a belief equal to $1$ on all states, 
whereas a given state $s$ is perfectly known 
if the belief is equal to $1$ on this state 
and to $0$ on all other states.
A new observation $o' \in \Omega$ 
updates the belief
with equation \ref{possBelUpdate},
easily derived from the possibilistic 
Bayes rule \cite{Dubois199023}: 
at time $t$, if the belief is $\beta_t$, then
\begin{eqnarray}
\label{possBelUpdate} \beta_{t+1}(s') & = &  \left \{ \begin{array}{ccc} 1 \mbox{ if } s' \mbox{ maximizes } \pi \paren{ o', s' \sachant \beta_t, a}; \\
 \pi \paren{ o', s' \sachant \beta_t, a} \mbox{ otherwise,}
\end{array} \right.
\end{eqnarray}
where $\pi \paren{o',s' \sachant \beta_t, a}
= \displaystyle \max_{s \in \mathcal{S}} \min \set{ 
\Omega^{\pi} \paren{ s',a, o'}, T^{\pi} \paren{ s,a,s'}, \beta_t(s) }$
is the joint possibility distribution over $\mathcal{O} \times \mathcal{S}$.
Then, if the current belief is $\beta$,
and selected action $a \in \mathcal{A}$, 
the next belief may be $\beta'$,
with a possibility $\pi \paren{ \beta' \sachant \beta,a}$
equal to the possibility of observations leading to it, 
computed from the joint possibility distibution: 
$\pi \paren{o' \sachant \beta,a} = \max_{s' \in \mathcal{S}} \pi \paren{o',s' \sachant \beta,a}$.
%$\pi \paren{o' \sachant \beta_t,a} = \max_{s' \in \mathcal{S}} \pi \paren{ o',s' \sachant \beta_t, a }$.
%The translation into $\pi$-MDP can be done in a similar way as for POMDPs: we denote by

Now, a belief state has a good preference 
when it is unlikely, according to it, that the system 
is in an unsatisfactory state: 
$\mu(\beta) = \min_{s \in \mathcal{S}} \max \set{ \mu(s), 1-\beta(s) }$. 
The possibilistic belief space which contains all the possibility 
distributions defined on $\mathcal{S}$ is denoted by $B^{\pi} \subsetneq \mathcal{L}^{\mathcal{S}}$.
A \textit{strategy} $\delta$ is defined 
as a sequence of functions 
$\delta_t: B^{\pi} \rightarrow \mathcal{A}$:
$a=\delta_t(\beta)$ is the action 
selected by the agent 
at step $t \in \mathbb{N}$.
%if it is following the strategy $(\delta_t)_{0 \leqslant t}$.

Let $h \in \mathbb{N}^*$ be the mission duration: 
the set of $h$-length trajectories
starting with the belief $\beta \in B^{\pi}$ 
by following strategy $\delta$ 
is denoted by $\mathcal{T}^{\delta}_h(\beta)$. 
The \textit{value function}  
measures the quality of a strategy $\delta$: 
for the initial belief $\beta_0 \in B^{\pi}$, 
it is the Sugeno integral of the preference
according to the possibility measure:
\begin{equation*}
\label{optimizationCriterion}
V_h^{\delta}(\beta_0) = \max_{ \mathcal{T}^{\delta}_h(\beta_0)} \min \set{ \min_{t=0}^{h-1} \pi \paren{ \beta_{t+1} \sachant \beta_{t}, \delta_{t}(\beta_{t})} , \mu(\beta_H) }.
\end{equation*}
The maximal value funtion 
and the associated optimal strategy 
can be computed 
using Dynamic Programming (DP) \cite{Sabbadin:1999:pipomdp}.
However, for concrete problems, 
the belief space can be dramatically large: 
$\# B^{\pi} = \# \mathcal{L}^{\# \mathcal{S}} - (\# \mathcal{L}-1)^{\# \mathcal{S}}$ 
explodes and the computations 
become intractable.
Moreover, a criterion without
the specification of $h$
is needed to cover missions 
with unbounded durations.

\section*{Qualitatif Possibilistic Mixed-Observable MDPs}
\label{QPMOMDP}

In practice, states are rarely totally hidden.
Like in \cite{OngShaoHsuWee-IJRR10}, 
we consider problems where the state space $\mathcal{S}$ 
can be written as a Cartesian product of 
a visible state space $\mathcal{S}_{v}$ 
and a hidden one $\mathcal{S}_h$: 
$\mathcal{S}$ = $\mathcal{S}_v$ $\times$ $\mathcal{S}_h$.
Let $s=(s_v,s_h)$ be a state of the system. 
The component $s_v$ is directly
seen by the agent 
and $s_h$ is partially observed 
through $o_h' \in \mathcal{O}_h$
whose possibility distribution 
is denoted by $\pi \paren{o_h' \sachant s',a }$.
Figure \ref{figureexp1}.a illustrates this structured model 
called \textit{Mixed-Observable} MDP ($\pi$-MOMDP). 
The next theorem leads to avoid 
some useless computations defining the space $\mathcal{X}$:
\begin{theorem}
\label{thm1} Each reachable belief of a $\pi$-MOMDP 
can be written as an element 
$x=(s_v,\beta_h)$ of $\mathcal{X} = \mathcal{S}_v \times B^{\pi}_h$ where $\beta_h \in B^{\pi}_h$ 
is a belief about $\mathcal{S}_h$.
\end{theorem}
The size of $B^{\pi}$ is $\# B^{\pi} \sim \# \mathcal{L}^{\# \mathcal{S}_v \cdot \# \mathcal{S}_h - 1}$.
The criterion is restricted to the space $\mathcal{X}$, whose size is exponentially smaller, 
$\# \mathcal{X} \sim \# \mathcal{S}_v \cdot \# \mathcal{L}^{\# \mathcal{S}_h-1}$:
\begin{equation} 
\label{optcriterion} V^{\delta}(x) = \max_{0 \leqslant h} \max_{ \tau \in \mathcal{T}^{\delta}_{h}(x)} \min \set{ \min_{t=0}^{h-1} \pi \paren{ x_{t+1} \sachant x_t, \delta_{t}(x_t) }, \mu(x_h) },
\end{equation}
where $\mu(x) = \mu(s_v,\beta_h) \hspace{-0.1cm} = \hspace{-0.15cm} \min_{s_h \in \mathcal{S}_h} \hspace{-0.1cm} \max  \hspace{-0.1cm} \set{  \hspace{-0.05cm} \mu(s_v,s_h), 1  \hspace{-0.1cm} -  \hspace{-0.1cm} \beta_h(s_h) \hspace{-0.05cm} }$ is the preference function.
As proved in \cite{Drougard13}, there exists an optimal strategy $\delta^{\ast}$, 
which is stationary \textit{i.e.} 
which does not depend on the stage of the process $t$. 
It can be found by the following Dynamic Programming (DP) scheme: initialization is $\forall x \in \mathcal{X}$, $V_0(x) = \mu(x)$, 
and until convergence, $ V_{i+1}(x) = \displaystyle \max_{a \in \mathcal{A}} \max_{x' \in \mathcal{X}} \min \set{ \pi \paren{ x' \sachant x , a} , V_i(x') }$,
\begin{eqnarray}
\label{DPE2} \mbox{ and if } V_{i+1}(x)>V_i(x), \hspace{0.1cm} \delta^*(x) = \operatorname*{argmax}_{a \in \mathcal{A}} \max_{x' \in \mathcal{X}} \min \set{ \pi \paren{ x' \sachant x , a} , V_i(x') }.
\end{eqnarray}
\begin{theorem}
\label{thmIV} If there exists a ``noop'' action $\overline{a}$ such that, 
$\forall x \in \mathcal{X}$, 
$\pi \paren{x' \sachant x, \overline{a} } = 1$ if $x'=x$ and $0$ otherwise, 
dynamic programming equations \ref{DPE2} compute the maximum 
criterion and an optimal policy.
This dynamic programming scheme converges in at most 
$\# \mathcal{X}$ iterations.
\end{theorem}
The hypothesis of the action $\overline{a}$ existence
is not a constraint in practice
as it is only selected in some goal states.
Note that the value function is defined regardless of the mission duration.

The following application shows the efficiency 
of the $\pi$-MOMDPs framework to provide strategies
well-suited for a recognition mission. 
Consider a robot on a grid $g \times g$, 
which perfectly knows its location $(x,y) \in \{ 1, \ldots, g \}^2$:
each location is a visible state $s_v \in \mathcal{S}_v$. 
It starts at location $s_{v,0}=(1,1)$, 
and knows that a target $T1$ is located 
in $(x_1,y_1)=(1,g)$, 
and a target $T2$ in $(x_2,y_2)=(g,1)$.
%and the robot perfectly knows their positions. 
One of the targets is $A$, the other $B$,
and the robot has to identify 
and reach target $A$ as soon as possible. 
The two situations, ``$T1$ is $A$'', and ``$T2$ is $A$'', 
constitute the hidden state space $\mathcal{S}_h$,
and the robot starts with the ignorant belief $(1,1)$:
both situations are possible.
Robot's actions $\mathcal{A}$
consist in moving in the four directions 
plus the action ``noop''.
Targets pictures are analyzed 
producing an observation 
of the targets' natures: 
$oAA$ if both targets are observed as beeing $A$,  
or $oAB$ if only $T1$, 
$oBA$ if only $T2$, and $oBB$ if noone.
The more the robot is far from a target, 
the more the observation is noisy. 
The highest reward and preference values
occur when the robot selects the noop action 
in the location of target $A$,
and lowest ones in target $B$ location.
Without the Mixed Observability exploitation, 
classical algorithms could not solve even 
very small $3 \times 3$ grids. 

After the strategies computations, 
we assume that in reality,
used computer vision algorithms 
badly perform when the robot is 
farther than $r$ from targets,
as illustrated in Figure \ref{figureexp1}.b.
Distant view pictures of targets 
are lacking in the training set, 
leading to a fixed misperception probability
$P_{bad} > \frac{1}{2}$.  
In other spots, the probability 
of an incorrect observation is lower than $\frac{1}{2}$
but increases with the distance 
from the target.  
We compare the averages of the rewards gathered at execution
by strategies from the probabilistic and possibilistic models.
If $k$ is the number of time steps to identify 
and reach the correct target,
the total reward is $100-k$,
as the reward function returns $-1$
until target $A$ is reached, 
rewarded by $100$.
Figure \ref{figureexp1}.c shows that 
the probabilistic model is more affected by the introduced error 
than the possibilistic one.
%it shows the total reward at execution of each model 
%as a function of $P_{bad}$, 
%the probability of an incorrect 
%target observation in the error zone
The possibilistic belief update (\ref{possBelUpdate})
is responsible as it only takes into account 
more reliable observations:
the next belief is either more skeptic 
about a state if its confirms the prior belief; 
or changes to the opposite belief if it contradicts it.
On the contrary a probabilistic belief is modified at each step. 


Consider $\pi$-MOMDPs 
whose space $\mathcal{X}$ is represented by
$n$ boolean variables $X=(X_1, \ldots, X_n)$.
As illustrated in Figure \ref{factADD}.a,
variables $X_i$ of a \textit{factorized $\pi$-MOMDP} 
are said post action independent as there are no
dependences (arrows) between variables of the same time step.
Resulting transition functions $T^a_i = T^{\pi} \paren{X_i' \sachant X,a}$,
such that $\min_{i} T_i^a = T^{\pi} \paren{X' \sachant X,a}$,
are encoded with \textit{Algebraic Decision Diagrams} 
(ADDs, see Figure \ref{factADD}.b). 
Algorithm PPUDD is a symbolic version of 
the Dynamic Programming scheme (\ref{DPE2}).
Inspired by SPUDD \cite{Hoey99spudd:stochastic}, 
PPUDD means \emph{Possibilistic Planning Using Decision Diagrams}. 
As SPUDD, it operates on ADDs 
encoding value, preference and transition functions.
That is why $\max$ and $\min$ operators 
are circled in algorithm PPUDD:
operations are directly performed 
between trees reducing the number of calculations.
While SPUDD uses operators $+$ and $\times$ 
and creates then news values 
leading potentially to huge ADDs, PPUDD only uses
operators $\min$ and $\max$, 
and the number of values is non increasing: 
ADDs are smaller, and thus computations are faster.
First equation of DP scheme (\ref{DPE2}) is divided into $n$ steps (see loop line $6$).
The second equation corresponds to line $10$. 
The implementation was performed 
with the \textit{CU Decision Diagram Package}
for the ADDs computations. 
A natural factorisation of $\mathcal{X}$ comes from some independence assumptions
illustrated by Figure \ref{figureexp2}.a. As proved in \cite{DBLP:conf/aaai/DrougardTFD14}, 
beliefs of a $\pi$-MOMDP satisfacing these assumptions
are post action independent beliefs: $x = (s_{v,1}, \ldots, s_{v,m}, \beta_1, \ldots, \beta_{l})$, 
with $\beta_i$ a belief over variable $s_{h,i}$.
The independence of sensors and of corresponding hidden states suffice to fullfil these conditions.
A problem known as 
the \textit{RockSample} problem, 
looks like the one 
of section \ref{QPMOMDP}.
It is more complex 
as the number of targets
is greater than two: 
targets are rocks, 
and the robot is expected 
to sample only ``scientifically good'' ones,
estimating their nature 
through specific observation actions.
As this problem respects assumptions 
encoded by the graphical 
representation in Figure \ref{figureexp2}.a, 
the model is naturally expressed 
as a factorized $\pi$-MOMDP. 
PPUDD performances are compared with 
the probabilistic MOMDP solver APPL \cite{Kurniawati08sarsop:efficient}
and the symbolic HSVI solver \cite{citeulike:2961378},
first in terms of computation time, 
and below with the average of 
the total reward at execution 
(see Figure \ref{figureexp2}.b). 
PPUDD computes a strategy maximizing 
exactly the possibilistic criterion 
(\ref{optimizationCriterion})
while APPL and symb. HSVI compute solutions
which approximately maximize the expected sum of rewards.
These experimental results show that using an exact algorithm 
(PPUDD) for an approximate model ($\pi$-MOMDPs) 
can run significantly faster 
than reasoning about exact models, 
while providing better policies 
than approximate algorithms (APPL) 
for exact models.

To focus on the behaviour 
of this possibilistic qualitative approach 
in a wide panel of probabilistic problems, 
we participated  in the fully observable track of the 
\textit{International Probabilistic Planning Competition 2014} 
(IPPC 2014\footnote{\url{https://cs.uwaterloo.ca/~mgrzes/IPPC_2014/}})
with PPUDD. 
Figure \ref{IPPC} shows performances of competitors
using ADDs: 
PPUDD and a probabilistic solver called symbolic LRTDP 
as a variant of \cite{Bonet03labeledrtdp:}.
\section*{Possibilistic Processes in Human-Machine Interaction}
\section*{An Hybrid POMDP}
\section*{Conclusion and perspectives}
In this paper, qualitative possibilistic models
for planning under uncertainty were presented,
studied to adress some issues 
of the probabilistic POMDPs.
While the probabilistic belief space is infinite,
the possibilistic one $B^{\pi}$ is finite:
it is a well-suited belief space discretization.
Moreover, total ignorance
and imprecision on observation probability distributions
are well-encoded with this formalism. 

However, as highlighted by the domain ``Traffic''
of the competition (see Figure \ref{IPPC}), the possibilistic criterion (\ref{optcriterion})
can be too optimistic.
Experiments on the left 
concerning the Navigation problem
is a good illustration: 
a robot which has to
reach a goal, but may disappear in some spots.
The classical criterion (\ref{optcriterion})
(M1, optimistic) 
leads to a poor strategy, 
while a cautious criterion (M2, pessimistic)
is more suited: 
for most of missions however 
the reverse occurs.
Proposed approaches involves then 
the choice of a criterion.
Note that in this example, 
SPUDD provides good strategies, but 
cannot solve big instances. 

Presented models 
assume also that preference
and uncertainty degrees
share the same scale
which makes it hard to
set in practice: the use of
the aggregation operator leximin 
is a good alternative,
but makes the problem more complex.
Finally, state space search algorithms
(PROST \cite{conf/aips/KellerE12} and GOURMAND) won IPPC 2014,
and are yet far more efficient than ADD-based methods.
A future work is motivated by these issues
and builds a POMDP with possibilistic beliefs in $B_h^{\pi}$,
rewards in $\mathbb{R}$ attached to them,
and probabilistic transition functions.
As ADDs are still efficient for preprocessing,
they are used to translate this hybrid
model into a classical Fully Observable MDP.
Efficient sampling MDP solvers such as PROST 
are then able to solve this simplified POMDP. 


Optimal strategies for presented POMDPs (purely qualitative, and hybrid)
are computable in finite time, at most exponential in the description of the problem.

(+ CONCLUSION DE CHAQUE PAPIER)
Nous avons observ\'e la difficult\'e du formalisme possibiliste qualitatif \`a la
description d'un probl\`eme de robotique autonome. Nous avons aussi confirm\'e
l'int\'er\^et d'utiliser ce formalisme dans la mod\'elisation de la croyance de l'agent.
De plus, les r\'esultats de la comp\'etition de planification incite \`a abandonner l'approche symbolique pour
une recherche dans l'espace d'\'etats. Pour cela, nous d\'eveloppons un mod\`ele hybride ...
\textbf{D\'eveloppement d'un mod\`ele hybride probabiliste et possibiliste}.


as in contingent/conformant planning \cite{Albore_atranslation-based},
more refined

